\documentclass[a4paper,12pt]{book}
\usepackage[margin=0.7in,nomarginpar]{geometry}
\usepackage[italian]{babel}
\usepackage[utf8]{inputenc}
\usepackage{graphicx}
\usepackage{mathtools}
\usepackage{amsfonts}
\usepackage{hyperref}
\usepackage{tabularx}
\usepackage{amsthm}
\usepackage{array}
\usepackage{xcolor}
\usepackage{polynom}
\usepackage{wrapfig}
\usepackage{stackengine}
\usepackage{float}
\usepackage{placeins}
\usepackage{tikz}
\usetikzlibrary{arrows,automata}
\restylefloat{table}
\graphicspath{ {./images/} }

\addtolength{\topmargin}{.2in}
\addtolength{\textheight}{-.3in}


% commands
\newcommand{\red}[1]{{\color{red}#1}}
\newcommand{\blue}[1]{{\color{blue}#1}}
\newcommand{\green}[1]{{\color{teal}#1}}
\newcommand{\cita}[1]{{\hfill \footnotesize #1 }}
\newcommand{\ov}[1]{\overline{#1}}
\newcommand{\un}[1]{\underline{#1}}
\newcommand{\gn}[1]{\lceil #1 \rceil}
\newcommand{\sq}{\square}
\newcommand{\nin}{\notin}
\newcommand{\MCD}{\text{MCD}}
\newcommand{\mcm}{\text{mcm}}
\newcommand{\V}{1}
\newcommand{\F}{0}

\newcommand\xrowht[2][0]{\addstackgap[.5\dimexpr#2\relax]{\vphantom{#1}}}

\DeclarePairedDelimiter\abs{\lvert}{\rvert}
\DeclarePairedDelimiter\ceil{\lceil}{\rceil}
\DeclarePairedDelimiter\floor{\lfloor}{\rfloor}

\newcommand{\hlinee}{\begin{tabular}{p{\textwidth}}\hline\ \\\end{tabular}}

\def\C{\mathbb{C}}
\def\N{\mathbb{N}}
\def\Q{\mathbb{Q}}
\def\R{\mathbb{R}}
\def\Z{\mathbb{Z}}

\def\geom{\text{Geom}}

\def\p{\prob}
\newcommand{\prob}[1]{
	\text{P}\left(#1\right)
}

\newcommand{\E}[1]{
	\mathbb{E}\left[#1\right]
}

\newcommand{\var}[1]{
	\text{Var}\left(#1\right)
}

\newcommand{\nb}[1]{
	\text{NB}\left(#1\right)
}

\newcommand{\parts}[1]{
	\mathcal{P}\left(#1\right)
}

% matrix commands
\newcommand{\vvec}[2]{
	\begin{bmatrix}
		#1 \\ #2
	\end{bmatrix}
}
\newcommand{\vvvec}[3]{
	\begin{bmatrix}
		#1 \\ #2 \\ #3
	\end{bmatrix}
}

\newcommand{\vvvvec}[4]{
	\begin{bmatrix}
		#1 \\ #2 \\ #3 \\ #4
	\end{bmatrix}
}

\def\enddim{
	\begin{flushright}
		$ \square $
	\end{flushright}
}

\theoremstyle{plain}% default
\newtheorem{thm}{Teorema}[section]
\newtheorem{lem}[thm]{Lemma}
\newtheorem{prop}[thm]{Proposizione}
\newtheorem*{cor}{Corollario}
\theoremstyle{definition}
\newtheorem{defn}{Definizione}[section]
\newtheorem{conj}{Congettura}[section]
\newtheorem{exmp}{Esempio}[section]
\newtheorem{exrc}[exmp]{Esercizio}
\theoremstyle{remark}
\newtheorem*{comm}{Commento}
\newtheorem*{note}{Nota}
\newtheorem{caso}{Caso}

\begin{document}

\author{Lezioni della prima parte di Marco Ghimenti\\Lezioni della seconda parte di Maurizio Pratelli\\ A cura di Alessandro Cheli}
\title{Appunti di Calcolo Probabilit\`a e Statistica}
\date{A.A 2019-2020}

\frontmatter
\maketitle
\tableofcontents

\mainmatter
\part{Probabilità Discreta e Catene di Markov\\Lezioni di Marco Ghimenti}


\chapter{Probabilit\`a Discreta e Condizionata}

\section{Probabilit\`a Discreta e Formule Combinatorie}

\begin{defn}
    Probabilit\`a: \textbf{Attendibilit\`a confortata da motivi ragionevoli}
\end{defn}
    La probabilit\`a (discreta) di un evento si pu\`o definire, in maniera intuitiva,
\begin{defn}
        \begin{equation}
            \prob{\text{evento}} = \dfrac{\text{\# casi favorevoli}}{\text{\# casi possibili}}
        \end{equation}

\end{defn}

\begin{exmp}
    Prendiamo ad esempio il lancio di un dado, voglio ottenere un numero $ \geq 5 $, la probabilit\`a dell'evento \`e

    \begin{equation*}
        P = 2/6 = 1/3
    \end{equation*}
    
    Un altro esempio pu\`o essere la probabilit\`a, lanciando 2 dadi, che almeno uno dei due renda un numero  $ \geq 4 $.
    
    \begin{equation*}
        P = \dfrac{27}{6^2} = \dfrac{3}{4}
    \end{equation*}
    
    I casi favorevoli sono 27 perch\'e lanciando se lanciando il primo dado ottenendo un numero $ \leq 3 $ significa che ho 3 possibili casi per ognuno dei lanci del primo dado per ottenere un numero $ \geq 4 $ dal lancio del secondo dado ($ 3 \cdot 3 $), a cui si aggiungono $ (3 \cdot 6) $ casi se ottengo un numero $ \geq 4 $ dal primo lancio (tutti i casi del secondo lancio sono validi.)
\end{exmp}

\begin{exmp}
    Qual \`e la probabilit\`a di ottenere almeno un asso pescando 2 carte da un mazzo di 54?

    \begin{equation*}
    P = \dfrac{(50 \cdot 4) + (53 \cdot 4)}{54 \cdot 53} = \dfrac{206}{1431}
    \end{equation*}
    
    Per i casi possibili, ho 54 casi per la prima pescata e 53 per la seconda, per i casi favorevoli ho
    
    \begin{equation*}
    \begin{cases}
        \text{Se pesco un Asso alla prima e una carta qualsiasi alla seconda} \implies 4 \cdot 53 \\
        \text{Se non pesco un Asso alla prima e un Asso alla seconda} \implies 50 \cdot 4
    \end{cases}
    \end{equation*}
    
\end{exmp}
\begin{exrc}
    Calcolare la probabilit\`a di pescare esattamente 2 assi pescando 5 carte
    \end{exrc}
    \begin{exrc}
    Calcolare la probabilit\`a di pescare esattamente 2 due donne pescando 5 carte
    sapendo che la prima carta uscita \`e una figura
    \end{exrc}
    
    Svolgere questi due esercizi contando i casi \`e molto macchinoso. Serve introdurre un po' di calcolo 
    combinatorio e, per il secondo esercizio, il concetto di probabilit\`a condizionata.
    
    \subsection{Principio di induzione}
    
    Vogliamo dimostrare una proposizione che dipende da un indice $n\in\mathbb{N}$. Una possibile strategia dimostrativa \`e il {\em principio di induzione}
    Se riusciamo a dimostrare che
    \begin{enumerate}
    \item La proposizione \`e verificata per un certo indice $n_0$
    \item Se assumiamo per verificata la proposizione per un generico indice $n$, allora riusciamo a dimostrare
    la proposizione per l'indice $n+1$
    \end{enumerate}
    allora la proposizione  \`e verificata per ogni $n\ge n_0$.

\subsection{Permutazioni di n elementi}

\begin{defn}
    Una permutazione \`e uno scambio dell'ordine di una sequenza di elementi che possono essere di qualunque tipo. L'obiettivo \`e trovare il numero di tutte le permutazioni (cio\`e tutte le sequenze con ordine) possibili dato un certo numero n di elementi.
\end{defn}

\begin{prop}
        Le permutazioni di un insieme di $ n $ elementi sono definite come
        \begin{equation}
        \text{Perm}(n) = n!
        \end{equation}
\end{prop}

\begin{proof}
    Dimostriamo per induzione. Come passo base possiamo verificare immediatamente che 
    $ \text{Perm}(1) = 1 $.

    Il passo induttivo sar\`a 
    \begin{equation*}
        \begin{aligned}
            \text{Perm}(n) = n! \implies \text{Perm}(n+1) = (n+1)! \\
            \text{Perm}(n+1) = (n+1) \cdot \text{Perm}(n) \\
            = (n+1) \cdot n! = (n+1)!
        \end{aligned}
    \end{equation*}
\end{proof}

\subsection{Coefficiente Binomiale}

\begin{defn}
    Il coefficiente binomiale \`e un numero intero non negativo definito dalla formula
    \begin{equation}
        \binom{n}{k} = \dfrac{n!}{(n-k)!k!}
    \end{equation}
\end{defn}

\begin{prop}
    Indicato con $S_{n,k} $ il numero di modi possibili di scegliere $k$ oggetti da un insieme di $n$ elementi
    vale $S_{n,k}=\binom{n}{k}$
\end{prop}

\begin{proof}
    Fissato $ k \geq 2 $ dimostriamo per induzione su $ n \geq k $
    
    Il primo passo iniziale \`e, per $ n = k $

    \begin{equation*}
        S_{k,k} = 1=\binom{k}{k}= \dfrac{k!}{k!(k-k)!}
    \end{equation*}
    (c'\`e un solo modo di prendere $k$ elementi da un insieme di $k$ oggetti: prenderli tutti)

    Per questa dimostrazione serve anche un ulteriore passo iniziale: dobbiamo vedere in quanti
    modi si possono scegliere $k-1$ elementi da un insieme di $k$ oggetti. Abbiamo
    \begin{equation*}
        S_{k,k-1}= k =\binom{k}{k-1}= \dfrac{k!}{(k-1)!(k-k+1)!}
    \end{equation*}
    (infatti dobbiamo semplicemente scegliere quale elemento non prendere, e quindi abbiamo
    $k$ scelte possibili)

    Passo induttivo: consideriamo che $S_{n+1,k} = S_{n,k} + S_{n,k-1}$ infatti posso scegliere i $k$
    elementi dai primi $n$, e scartare l'ultimo, oppure sceglierne $k-1$ dai primi $n$ e prendere l'ultimo.
    Questa formula spiega perch\'e abbiamo bisogno di due passi iniziali.

    \begin{eqnarray*}
        S_{n+1,k} &=& S_{n,k} + S_{n,k-1}= \binom{n}{k}+\binom{n}{k-1}\\
                &= &\dfrac{n!}{k!(n-k)!} + \dfrac{n!}{(k-1)!(n-k+1)!} \\
                &= &\dfrac{n!((n-k+1) + k)}{k!(n-k+1)!} = \dfrac{n!(n+1)}{k!(n+1-k)!} = \dfrac{(n+1)!}{k!(n+1-k)!} \\
                &= &\binom{n+1}{k}
    \end{eqnarray*}
\end{proof}

\subsection{Disposizioni}

\begin{defn}
    Una disposizione $ D_{n,k} $ significa il numero di modi per "prendere" $ k $ oggetti ordinati da un insieme di $ n $ elementi.
\end{defn}
Ovviamente avremo
    \begin{equation}
    D_{n,k} = S_{n,k} \cdot \text{Perm}(k) = \dfrac{n!}{(n-k)!k!} \cdot k! = \dfrac{n!}{(n-k)!}
    \end{equation}

\section{Probabilit\`a Condizionata}

\begin{exmp}
    Lancio due dadi sommando il risultato, qual \`e $ \prob{\geq 10} $ sapendo che il primo ha fatto almeno 3?

    Sappiamo che $ \prob{\text{Somma} \geq 10} = 6/36 = 1/6 $
    
	Poniamo il vincolo che il lancio del primo dado risulti almeno $ \geq 3 $. Allora se vediamo i possibili
    risultati vediamo che i casi favorevoli sono $6$ e quelli possibili sono $24$, quindi

    \begin{equation*}
    \prob{\text{Somma} \geq 10 \mid \text{Primo dado} \geq 3} = 6/24 = 1/4
    \end{equation*}
\end{exmp}



\begin{defn}
    
    Ponendo $ \Omega = $ gli eventi possibili;
    La \textbf{probabilit\`a condizionata} che succeda $ A $ sapendo $ B $ si indica con:
    
    \begin{equation}
    \begin{aligned}
    \prob{A|B} = \dfrac{\text{casi favorevoli}}{\text{casi possibili}} = \dfrac{\abs{A \cap B}}{\abs{B}} = \dfrac{\abs{A \cap B}}{\abs{\Omega}} \cdot \dfrac{\abs{\Omega}}{\abs{B}} \\
    \prob{A|B} = \dfrac{\prob{A \cap B}}{\prob{B}}	
    \end{aligned}
    \end{equation}
    
\end{defn}

\begin{exmp}
    Nel lancio di un dado, la probabilit\`a di ottenere $ \leq 4 $ sapendo che \`e uscito un numero pari \`e
    
    \begin{equation*}
    \begin{aligned}
    \prob{\leq 4|\text{pari}} = \dfrac{\prob{\leq 4 | \text{pari}}}{\prob{\text{pari}}} \\
    \prob{\text{pari}} = 3/6 = 1/2 \\
    \prob{\leq 4 \cap \text{pari}} = 2/6 \\
    \implies \prob{\leq 4|\text{pari}} = \dfrac{2/6}{1/2} = 2/3
    \end{aligned}
    \end{equation*}
\end{exmp}

\begin{defn}
    Definiamo il \textbf{complementare} di un evento, ovvero $ A^C=\Omega \backslash A $. La probabilit\`a di un complementare \`e $ \prob{A^C} = 1 - \prob{A} = \prob{\Omega} - \prob{A} $
\end{defn}
In generale, dati due eventi $ A, B $ con $ A \cap B \neq 0 $ si ha che la probabilit\`a dell'unione \`e $ \prob{A \cup B} = \prob{A} + \prob{B} $.


\begin{defn}
Due eventi $A$ e $B$ si dicono  \textbf{indipendenti} se $\prob{A \mid B}=\prob{A}$.
\end{defn}

Dalla definizione di probabilit\`a condizionata si ottiene che se $A$ e $B$ sono indipendenti vale
 $ \prob{A \cap B} = \prob{A} \cdot \prob{B} $


\section{Esercizi}

\begin{exrc}
    \textbf{Terno al lotto}: Giocando 5 numeri al lotto (estrazione da 1 a 90) calcolare la probabilit\`a di ottenere un terno esatto e pi\`u di un terno.
    
    Se vogliamo ottenere un terno esatto i casi possibili sono $ \binom{90}{5} $ (I modi di estrarre 5 palline dall'urna). I casi favorevoli saranno $ S_{5,3} \cdot S_{85,2} = \binom{5}{3} \cdot \binom{85}{2}$
    (ovvero i modi in cui si possono scegliere $3$ numeri tra i $5$ estratti e $2$ numeri dagli altri).

    La probabilit\`a di ottenere un terno esatto sar\`a quindi
    \begin{equation*}
        \prob{\text{terno esatto}} = \dfrac{\binom{5}{3} \cdot \binom{85}{2}}{\binom{90}{5}} \simeq \frac 1{1230}
    \end{equation*}


    Per ottenere almeno un terno i casi favorevoli sono
    \begin{itemize}
        \item terno: $ \binom{5}{3} \binom{85}{2} $
        \item quaterna: $ \binom{5}{4} \binom{85}{1} $
        \item cinquina: 1
    \end{itemize}
    
    La probabilit\`a di ottenere almeno un terno sar\`a data dalla somma delle probabilit\`a corrispondenti a terno, quaterna e cinquina:
    
    \begin{equation*}
        \prob{\text{almeno un terno}} = \dfrac{\binom{85}{2} \binom{5}{3}}{\binom{90}{5}} + \dfrac{\binom{85}{1} \binom{5}{4}}{\binom{90}{5}} + 1
    \end{equation*}
\end{exrc}

\begin{exrc}
    \textbf{Probabilit\`a del gioco di Monty Hall}
    Nel gioco televisivo di Monty Hall il partecipante deve scegliere una fra tre porte, una di esse contiene un premio mentre le altre due contengono rispettivamente due capre. Dopo la scelta del giocatore iniziale il presentatore apre una delle due porte contenenti una capra. Al giocatore conviene cambiare porta o mantenere quella scelta in origine?
    
    \paragraph{Ipotesi} Se scelgo una porta e la mantengo vinco solo se il premio era nella porta che ho scelto $ \implies P = 1/3 $
    
    \paragraph{Ipotesi} Se scelgo una porta e la cambio avr\`o $ P = 2/3 $
    \begin{table}[H]
        \centering
        \caption{Gioco di Monty Hall}
        \begin{tabular}{lll}
            1  & 2  & 3  \\
            x  & x  & \$ \\
            x  & \$ & x  \\
            \$ & x  & x 
        \end{tabular}
    \end{table}
\end{exrc}
 
\begin{exrc}
    \textbf{Dado rosso e dado nero}
    
	Tiriamo due dadi, uno rosso ed uno nero. Calcolare la probabilit\`a che il dado rosso risulti 3 sapendo che sul dado nero \`e uscito 2:

    \begin{equation*}
        \begin{aligned}
            \prob{R=3 \mid N=2} = \dfrac{\prob{R=3 \cap N=2}}{\prob{N=2}} = 	\dfrac{1/36}{6/36} = \dfrac{1}{6} \\
            \prob{R=3} = 1/6
        \end{aligned}
    \end{equation*}
    Ne otteniamo che $ \prob{A|B} = \prob{A} \implies A, B $ sono indipendenti.
\end{exrc}


\chapter{Spazio Probabilizzato}

Uno spazio probabilizzato è definito come

\[ (\Omega, F, \mathcal{P}) \]

$ \Omega $ è l'insieme degli eventi elementari, ad esempio in un lancio di un dado $ \Omega = \{1,2,3,4,5,6\} $

$ \mathcal{P}(A)$, ovvero le parti di A, sono tutti gli insiemi che posso costruire a partire dagli elementi di A. Ad esempio:
\begin{align*}
& A = \{0, 1\} \\
& \mathcal{P}(A) = \{0, \{0, 1\}, \{0\}, \{1\}\}
\end{align*}

$ F \subseteq \mathcal{P}(\Omega) $ è un sottoinsieme delle parti di omega, chiuso rispetto a intersezione, unione e complementare.
\[ A,B \in F \implies \begin{cases}
A \cup B \\
A \cap B \\
A^C, B^C
\end{cases} \in F\]

Se $ A_1, \dots, A_n \subset F \implies \bigcup\limits_{n \in \N} A_n \in F $

Se $ F $ comprende queste proprietà si dice che è una $ \sigma $ Algebra (tribù)

La probabilità $ P $ è una funzione definita come

\begin{align*}
&P : F \to [0,1] \\
& P(\Omega) = 1 \\
& \forall i \neq j . A_i \cap A_j \neq 0 \implies P \left( \bigcup\limits_{n} A_n \right) = \sum_{n} P(A_n)
\end{align*}

Dato $ \Omega $ insieme finito, allora $ F = \mathcal{P}(\Omega) $ allora la probabilità sarà
\[  P (A \subset F) = \dfrac{\#A}{\#\Omega} \]

\paragraph{Esempio: Lotteria di De' Finetti}
Si assiste all'estrazione di un numero $ n \in \N $ casuale. Supponiamo che ogni numero abbia la stessa probabilità di essere estratto.

Dato un altro naturale $ m \in \N $

\begin{align*}
P_n = P(n) = \text{ La probabilità di estrarre il numero n}
\end{align*}

Vogliamo che 

\begin{align*}
& 0 \leq P_n \leq 1 \\
& P_m = P_n \forall n = m
\end{align*}

Quindi $ 1 = P(\Omega) = P \left( \bigcup\limits_{n \in \N} n \right) = \sum_{n \in \N} P_n $

Assumendo $ P_n = 0, \forall n $ allora $ \sum_{n} p_n = 0 $

Se $ P_n = c > 0, \forall n $ allora $ \sum_{n} p_n = \sum_{n} c = + \infty $

Ciò significa che nella lotteria di De' Filetti è impossibile che ogni numero sia equiprobabile.

\paragraph{Dimostrazione}

La terna $ (\Omega, F, \mathcal{P}) $ si può dimostrare.

\[ P(A^C) = 1 - P(A) \impliedby \begin{cases}
A \cap A^C \neq 0 \\
A \cup A^C = \Omega \\
P(\Omega) = 1
\end{cases} \]

\[ P(A \cup B) = P(A) + P(B) - P(A \cap B)\]

\[ A \subseteq B \implies P(A) \leq P(B) \]

\paragraph{Densità di Probabilità}

Definiamo $ \{p_n\}_{n \in \N} $ con $ p_n \in \R $, detta densità di probabilità come $ p_n \geq 0 $ e $ \sum_{n \in \N} p_n = 1 $ 

Se $ \{p_n\} $ è una densità di probabilità discreta $ \implies \left( \N, \mathcal{P}(\N), P\right) $ è uno spazio probabilizzato. Vale anche per eventi non equiprobabili.

\paragraph{Definizione, indipendenza degli eventi} 

$ P(A|B) \equiv \dfrac{P(A \cap B)}{P(B)} $ 

Allora $ A,B $ indipendenti se $ P(A|B) = P(A) $

È vero quindi che $ P(A|B) = P(A) \implies P(B|A) = P(B) $? Sì se $ A,B $ sono indipendenti.

\[ P(A|B) = \dfrac{P(A \cap B)}{P(B)} = P(A) \]
\[ P(B|A) = \dfrac{P(B \cap A)}{P(A)} = P(B) \]
\[ P(A) = \dfrac{P(A \cap B)}{P(B)} \]
\[ P(B|A) = \dfrac{P(B \cap A)}{P(A)} = \dfrac{P(B) \cdot P(A)}{P(A)} = P(B)\]
\[ P(A \cap B)  = P(A) \cdot P(B) = P(B \cap A) \]

Se $ A_1, \dots, A_n $ sono indipendenti:
\[ \forall i_1, \dots, i_k \land k \leq n \] (Con indici tutti diversi) allora vale \[ P(A)\cap \left(A_i1,\dots,A_ik\right) = P(A_i1) \cdot \dots \cdot P(A_ik) \] 

\paragraph{Esercizio teorico}

$ A, B $ indipendenti $ \implies \begin{cases}
A, B^C \\
A^C, B^C \\
A^C, B
\end{cases} $ indipendenti

\textbf{Dimostrazione:}
\[ A = (A \cap B) \cup (A \cap B^C)  \]
\[ P(A) = P(A \cap B) + P(A \cap B^C) = P(A) \cdot P(B) + P(A \cap B^C) \]
\[ P(A \cap B^C) = P(A) - P(A) \cdot P(B) = P(A)\left[1 - P(B)\right] = P(A) \cdot P(B^C) \]

$ A $ e $ B^C $ sono indipendenti \enddim
\chapter{Variabili Aleatorie}

\begin{figure}[H]
	\centering
	\caption{Tipi di variabili casuali}
	\includegraphics[]{varrandom}
\end{figure}

\section{Variabili Aleatorie Discrete}

Una variabile aleatoria (casuale o stocastica) discreta è una variabile che può assumere diversi valori in dipendenza da qualche fenomeno casuale.
Il risultato del lancio di un dado, ad esempio, è una variabile aleatoria discreta.

Prendiamo uno \textbf{spazio probabilizzabile} $ (\Omega, F) $ e una variabile aleatoria discreta $ x : \Omega \to \R $, che è una funzione continua non surgettiva. I valori di $ x $ devono essere un sottoinsieme finito di $ \R $ ovvero $ \{a_1, \dots, a_k\} $.
Vogliamo anche che $ \forall j \in [1, k] $ sia vero $ x^{-1}(a_j) = \{ \omega \in \Omega, x(\omega) = a_j\} \in F $. Utilizziamo la funzione inversa per ottenere gli elementi di $ \Omega $ su cui possiamo definire la probabilità.

Con la probabilità di tutti gli eventi definisco la densità di probabilità. Nello spazio probabilizzato $ (\Omega, F, \mathcal{P}) $ la probabilità che la variabile aleatoria assuma il valore $ a_j $ sarà $ p_j = \p{x = a_j} = \p{x^{-1}(a_j)} $, che viene detta densità di probabilità. Varranno quindi le seguenti proprietà

\begin{enumerate}
	\item $ \forall j \ldotp x^{-1} (a_j) $ sono tutti insiemi disgiunti.
	\item Essi coprono tutto $ \Omega $
\end{enumerate}

Vale che $ \sum_{j=1}^{k} p_j = 1 $. Poiché:
\[ 1 = \p{\Omega} = \p{\bigcup_{j} x^{-1}(a_j)} = \sum_{j} \p{x^{-1}(a_j)}\] 

Sia $ x : \Omega \to \{ a_1, \dots, a_k \} $ una variabile aleatoria, e sia la densità di probabilità $ p_j \geq 0 $ e anche  $ \sum_{j=1}^{k} p_j = 1 $, allora $ \p{x = a_j} = p_j $.

Preso uno spazio probabilizzabile $ (\Omega, F) $ e una variabile aleatoria $ x : \Omega \to \{ a_1, \dots, a_k \} $, supponendo che i numeri $ a_j $ siano ordinati, sia $ p_j $ la densità di probabilità, come posso ricostruire, ad esempio $ \p{x \leq a_3} $?

\[ \p{x \leq a_3} = \p{(x=a_1) \cup (x=a_2) \cup (x=a_3)} = \p{x = a_1} + \p{x = a_2} + \p{x = a_3}  \]

\paragraph{Esempio}

Voglio contare quanti 6 escono in 10 lanci di dadi.

Sia $ \Omega = \{ (1, 2, 3, 4, 5, 6)\}^{10} $, ovvero tutte le possibili parole di 10 elementi composte dai numeri da 1 a 6. Ad ogni lancio, ho $ \frac{1}{6} $ di probabilità di ottenere 6 e $ \frac{5}{6} $ di ottenere gli altri numeri. Definiamo la variabile aleatoria $ x : \Omega \to \{ 0, 1, 2, \dots, 9, 10 \} $ come il conteggio dei risultati dei lanci dove ottengo 6. Qual'è la probabilità di ottenere 3 lanci dove ho fatto 6?

\[ \p{x = 3} = \binom{10}{3} \left(\dfrac{1}{6}\right)^3 \left(\dfrac{5}{6}\right)^7 \]

% TODO spiega

\subsection{Legge di Bernoulli}
Faccio un esperimento, il risultato positivo ha probabilità $ p $, mentre il risultato negativo ha probabilità $ 1 - p $

Sia lo spazio $ \Omega = \begin{cases}
\text{successo} \to 1 \\
\text{insuccesso} \to 0
\end{cases} $

Una variabile aleatoria Bernoulliana è definita come $ x : \Omega \to \begin{cases}
p_1 = p \\ p_0 = 1 - p
\end{cases} $

\paragraph{Legge Binomiale} 
Sia $ k $ il conteggio dei successi di $ n $ esperimenti, abbiamo quindi che $ B(n,p) = x_i\{(0,1)\}^n \to \{0, \dots, n\} $. Abbiamo che la densità di probabilità Binomiale $ p_k = \p{x=k} = \binom{n}{k} p^k(1-p)^{n-k} $.

$ p_k $ è una densità? Sappiamo che $ p_k \geq 0 $ e sappiamo che $ 1 = \sum_{k} p_k $, con il binomio di Newton possiamo dimostrare che $ (a+b)^n = \sum_{k=0}^{n} \binom{n}{k} a^k b^{n-k} $. Proseguendo, abbiamo che 

\[ \sum_{k} p_k = \sum_{k=0}^{n} \binom{n}{k} p^k (1-p)^{n-k} = (p+(1-p)^n) = 1^n = 1 \]

\subsection{Somma di Variabili Aleatorie}
Lancio due dadi, uno rosso e uno nero, avremo quindi $ \Omega = \{R, N\} = \{(1,6)\}^2 $. Definisco due variabili aleatorie, $ x $ per il dado rosso dove $ x : (R, N) \to R $ e la variabile $ y : (R, N) \to N $. La densità per $ x $ sarà $ p_j = \frac{1}{6} \forall j $ mentre la densità per $ y $ sarà $ q_j = \frac{1}{6} \forall j $
Avremo che $ z = x + y $ conta la somma dei dadi.

\paragraph{Esercizi}
\begin{enumerate}
	\item Calcolare la densità di $ z $
		
		In questo caso $ x, y $ sono indipendenti, quindi avremo la densità di $ z $ detta $ t_z $.
		
		\[ \forall n \in [2, 12] .  \left( \p{z = n} = \sum_{i=1}^{n-1} \p{x = i} \cdot \p{y = n - i} \right) \]
		
		\begin{table}[H]
			\centering
			\caption{Distribuzione discreta della somma del lancio di due dadi.}
			\label{tab:distribdice1}
			\begin{tabular}{|l|l|l|l|l|l|l|l|l|l|l|l|}
				\hline
				$ x + y $     & 2        & 3        & 4        & 5        & 6        & 7        & 8        & 9        & 10       & 11       & 12       \\ \hline
				$ p_j + q_j $ & $ 1/36 $ & $ 2/36 $ & $ 3/36 $ & $ 4/36 $ & $ 5/36 $ & $ 6/36 $ & $ 5/36 $ & $ 4/36 $ & $ 3/36 $ & $ 2/36 $ & $ 1/36 $ \\ \hline
			\end{tabular}
		\end{table}
		
	\item Calcolare $ \p{4 \leq z \leq 6} $
	
	$ \p{4 \leq z \leq 6} = \p{4} + \p{5} + \p{6} = \frac{3}{36} + \frac{4}{36} + \frac{5}{36} = \frac{12}{36} = \frac{1}{3} $
\end{enumerate}

\begin{figure}[H]
	\centering
	\caption{Distribuzione della somma del lancio di due dadi}
	\includegraphics[]{binomdadi}
\end{figure}

% TODO Svolgi

\subsection{Indipendenza di Variabili Aleatorie}

Due variabili aleatorie $ x_1, x_2 $ sono indipendenti se $ \forall I_1, I_2, \subseteq \R $ intervalli o semirette si ha che 

\[ \left( \p{x_1 \in I_1} \cap \p{x_2 \in I_2} \right) = \p{x_1 \in I_1} - \p{x_2 \in I_2} \]

Nell'esempio di prima $ x, z $ e $ y,z $ sono dipendenti perché dati $ I_1 = [1,2], I_2 = [3,4] $ allora si ha che $ \p{x\in I_1} = \p{x=1,2} = \frac{1}{3} $ e si ha anche $ \p{z \in I_2} = \p{2 = 3,4} = \frac{5}{36} $. Ne otteniamo che:

% TODO fix erratum

\[ \p{(x \in I_1) \cap (z = 3,4)} = \p{(x = 1,2, z = 3,4)} = \dfrac{4}{12} \]

\paragraph{Variabili Aleatorie Congiunte}
Due variabili aleatorie $ x,y : \Omega \to \R $ discrete sono congiunte quando si può calcolare $ \p{x = m \cap y = n} = p_{n,m} $, ovvero una densità di probabilità $ \{ p_{n,m} \}_{n,m} $ con $ (p_{n,m} \geq 0 ) \land (\sum_{n,m} p_{n,m} = 1) $.

Sapendo $ p_{n,m} $ ricavo tutti i $ \p{x=m} = p_m $ e $ \p{y = n} = q_n $ con

\[ \p{x=m} = \p{x = m \cap \left\{ \bigcup_{n} y=n \right\}} = \sum_{n} \p{x=m,y=n} = \sum_n p_{n,m} = p_m \]
\[ \p{y=n} = \sum_{m} p_{n,m} = q_n \]

Non si può ricostruire dalle due probabilità la variabile aleatoria congiunta. Ad esempio, conoscendo $ p_n,q_n $ cerco $ p_n,m  = \p{x = m \cap y = n} = \p{x = m | y = n} \cdot \p{y=n} $. Se $ x,y $ sono indipendenti allora $ P(x=m|y=n) = \p{x=m} $ e vale il prodotto $ p_{m,n} = p_n \cdot p_m $.

\[ \sum_{m}\sum_{n} p_{n,m} = \sum_{m}\sum_{n} p_n \cdot q_n = \left( \sum_{m} p_m \right) \left( \sum_{n} q_m \right) = 1 \]

\paragraph{Formula di Convoluzione}

Tornando alla somma di due variabili aleatorie discrete, dati $ x,y $ indipendenti e $ z = x + y $, con $ x,y : \Omega \to \N $, dobbiamo calcolare $ \p{z = n} $ e la densità di probabilità discreta $ \left\{z = n\right\} $

\[ \left\{z = n\right\} = \bigcup_{i=0}^{n} \left\{x = 1 \cap y = n - i \right\} \]

\[ \p{z=n} = \sum_{i} \p{x = i \cap y = n-1} = \sum_{i = n}^{n} \p{x=i} \cdot \p{y=n-i} \]

\paragraph{Teorema: Rapporto fra Bernoulli e Binomiale}

Sommando $ n $ esperimenti di $ p $ dove conto i successi ottengo la binomiale $ B(n,p) $, quindi $ \sum_{i=1}^{n} \text{Bern}_i(p) = B(n,p) $

\paragraph{Dimostrazione per Induzione}
Caso base, per $ n = 1 \implies B(1,p) = \text{Bern}(p)$. Come passo induttivo abbiamo 
\[ B(n,p) = \sum_{i=1}^{n} \text{Bern}_i(p) \implies B(n+1,p) = \sum_{i=1}^{n+1} = \text{Bern}_i(p) \]
\[ \sum_{i=1}^{n+1} \text{Bern}_i(p) = \left( \sum_{i=1}^{n} \text{Bern}_i(p) \right) + \text{Bern}_{n+1}(p) = B)(n,p) + \text{Bern}(p) \]

Introduciamo le densità per continuare la dimostrazione
\[ \p{B(n,p) + \text{Bern}(p) = k} = \p{B(n,p) = k \cap \text{Bern}(p) = 0} + \p{B(n,p) = k-1 \cap \text{Bern}(p) = 1} \]
\[ = \binom{n}{k-1}p^{k-1}(1-p)^{n-k+1} \cdot p + \binom{n}{k}p^k(1-p)^{n-k} \cdot  (1-p) \]
\[ = \left[\binom{n}{k-1}\binom{n}{k}\right] p^k (1-p)^{n-k+1} = \binom{n+1}{k} p^k (1-p)^{n+1-k} = \p{B(n+1,p) = k}\]

\section{Variabile Geometrica}
Data una successione di esperimenti ripetuti con probabilità di successo $ 0 \leq p \leq 1 $. Lo ripeto fino ad ottenere un successo. $ \geom(p) $ conta il numero di prove necessarie. Ovvero $ \p{\geom(p) = k} = $ la probabilità di fare k esperimenti ed avere un successo dall'ultimo. La densità di probabilità sarà $ p_k = \p{\geom(p) = k} = (1-p)^{k-1} \cdot p $ con $ p_k \geq 0 $ e anche $ \sum_{k=1}^{\infty} (1-p)^{k-1} p = p \sum_{k=1}^{\infty}(1-p)^{k-1} $, che è una serie geometrica. Sarà quindi equivalente a $ p \sum_{i=0}^{n} (1-p)^i = \dfrac{p}{1-(1-p)} = 1$


Osservazione: consideriamo la serie geometrica $ \sum_{i=0}^{\infty} q^i $.
Sappiamo che $ (1-q) \sum_{i=0}^{\infty} q^i = 1 $ perché $ (1-q)(1+q+q^2+q^3\dots) $ = $( 1 - q + q - q^2 + q^2 + \dots )$, semplificando i termini rimane 1.

Una variabile geometrica \textbf{non ha memoria}, l'esperimento numero $ n $ ha la stessa probabilità degli altri esperimenti:

\[ \p{\geom(p) = n+m | \geom(p) > n} = \p{\geom(p) = m}  = \]
\[ = \dfrac{\p{\geom(p) = m + n \cap \geom(p) > n}}{\p{\geom(p) > n}} = \dfrac{\p{\geom(p) = m+n}}{\p{\geom(p) > n}}\] 
\[= \dfrac{(1-p)^{m+n-1} \cdot p}{(1-p)^n} = \p{\geom(p) = m} \]

Si può dimostrare sapendo che $ \forall m . \p{\geom(p) = n+m \cap \p{\geom} > n} = \p{\geom(p) = m+n} $

\subsection{Variabili Ipergeometriche}

Siano dati $ r $ sfere rosse, $ b $ sfere bianche, $ n $ estrazioni senza reimbussolamento, $ k = $ numero di sfere rosse estratte, $ H(b+r, r, n) $ conta il numero di sfere rosse estratte dopo n tentativi. Sappiamo che $ (0 < n \leq b + r) \land (k \leq n)  \land (k \leq r) \land (n-k \leq b) $.

Definiamo la densità di probabilità
\[ \p{H(b+r, r, n) = k} = \dfrac{\binom{r}{k} \binom{b}{n-k}}{\binom{b+r}{n}} \] 

\paragraph{Esercizio} Dati $ r = 10, 15 = b, 7 = n $ abbiamo che $ H(25,15,7) $

Esercizio per casa: $ \p{H(25, 15, 7) = 3} $
\chapter{Catene di Markov}

\section{Catene di Markov e Processi Stocastici}

\begin{defn}
	\textbf{Processi Stocastici}
	Spesso abbiamo bisogno di rappresentare quantit\`a incerte che cambiano nel tempo. Possiamo rappresentarle con famiglie di variabili aleatore indicizzate mediante un parametro, spesso corrispondente al ``tempo"

	Una famiglia di variabili aleatorie $ \left\{X_t\right\}_{t\in \mathcal{T}} $ dove $ \mathcal{T} \subseteq \R $ e che assumono tutte valori nello stesso insieme $ E $ \`e  detta \textbf{processo stocastico}. L'insieme $ E $ \`e  detto spazio degli stati del processo, mentre l'insieme $ \mathcal{T} $ \`e  detto insieme dei tempi. Considereremo sempre gli insiemi degli stati e dei tempi \textit{discreti} (numerabili) e molto spesso finiti. L'insieme dei tempi pu\`o  essere un intervallo $ \mathcal{T} = [0,T] $. Ad esempio, insiemi $ \mathcal{T} $ validi possono essere: $ \N, \Z, \{0, 1, 2, \hdots, n\}, \{t_1, t_2, \hdots, t_n\} $. Dato un processo stocastico $ \{X_t\}_{t \in \mathcal{T}} $ le variabili aleatorie $ X_t \in E $ sono dette marginali del processo. Le leggi delle marginali di due processi potrebbero coincidere, pur essendo i due processi molto diversi.

\end{defn}


\begin{exmp}\label{ex-fond}
Consideriamo le estrazioni da un urna contenente 10 palline rosse (R) e 3 palline blu (B). Prendiamo in considerazione il colore della pallina alla prima, seconda, terza, ecc. estrazione. Il fenomeno \`e  rappresentabile con una famiglia di variabili aleatorie.
	\begin{equation*}
	X_1, X_2, X_3, \hdots, X_n \in \{\text{rossa}, \text{blu}\}
	\end{equation*}
Considero tre tipi diversi di estrazione
\begin{enumerate}
\item estraggo senza reimbussolare
\item estraggo e reimbussolo la pallina estratta
\item estraggo la prima, estraggo la seconda e reimbussolo la prima, estraggo la terza e reimbussolo la
seconda, ecc. In pratica ad ogni estrazione nell'urna ci sono tutte le palline escluso l'ultima estratta
\end{enumerate}
Calcolare, nei tre casi, $\p{\text{3a estratta =R} | \text{1a estratta =B, 2a estratta=R} }$ nei tre casi.

Nel caso a) abbiamo  $\p{\text{3a =R} | \text{1a=B, 2a=R} }=\p{\text{3a =R}}=10/3$;

nel caso b)
 $\p{\text{3a =R} | \text{1a=B, 2a=R} }=9/11$;

 nel caso c)
 $\p{\text{3a =R} | \text{1a=B, 2a=R} }=\p{\text{3a =R} | \text{2a=R} }= 9/12 $
\end{exmp}

In tutti i tre casi ogni marginale $X_t$ si trova nello stesso modo:
 $$\p{X_t=R}=\frac{\text{palline rosse presenti nell'urna}}{\text{palline totali presenti nell'urna}}$$  ma la legge specifica ottenuta dipende dall'istante $t$ in cui si fa l'estrazione e - a seconda del tipo di processo - dalla storia delle estrazioni precedenti.

\begin{defn}
	\textbf{Processo di Markov:}
	Un processo aleatorio \`e detto di Markov  se l'evento all'istante $k+1$ dipende solo dall'esito dell'evento
	$k$ ma non da quelli precedenti. In formule, se $e_k$ rappresenta il valore assunto dalla variabile $X_k$, il processo $\{X_t\}_{t\in \N}$ \`e di Markov se
	$$\p{X_{k+1}=e_{k+1} | X_k=e_k\cap X_{k-1}=e_{k-1}\cap\dots\cap X_0=e_0}=\p{X_{k+1}=e_{k+1} | X_k=e_k}.
	$$
In altri termini si pu\`o anche dire
	\begin{quote}
		{Un processo \`e  di Markov se conoscendo il presente, passato e futuro sono indipendenti.}
	\end{quote}
\end{defn}
	\begin{exmp}
	Tornando all'esempio \ref{ex-fond} dell'estrazione dall'urna  un'estrazione con reimmissione \`e  sicuramente un processo di Markov, se l'estrazione \`e  senza reimmissione il processo \textbf{non} \`e  di Markov. Il motivo \`e  che tutta la sequenza di palline estratte \`e  necessaria per conoscere il contenuto esatto dell'urna (l'informazione passata non pu\`o  essere trascurata). Anche nel terzo caso si ha un processo di Markox, perch\'e le estrazioni non sono indipendenti tra di loro, ma ogni estrazione dipende solo dalla precedente.
\end{exmp}

\begin{note} Se un processo \`e di Markov per conoscerlo interamente bastano le probabilit\`a di
transizione $\p{X_{k+1}=j\ |\ X_{k}=i}$ (oltre alla marginale $X_0$).
\end{note}



\begin{defn}
	Un process di Markov $ {X_i}_{i\in \N} $ \`e  \textbf{omogeneo} se le probabilit\`a di transizione non dipendono dall'istante $k$, ovvero:
	\begin{equation*}
	\begin{aligned}
		\forall i,j \in \N\ ,\  \p{X_{k+1} = j \mid \{X_k = i\}} = \p{X_1 = j \mid  \{X_0 = i\}}
	\end{aligned}
	\end{equation*}
\end{defn}

\begin{exmp}
	Nell'esempio \ref{ex-fond}, a) e c) sono processi di Markov omogenei.
\end{exmp}

\begin{exmp}
	Modifichiamo l'esempoi  \ref{ex-fond} nel modo seguente: oltre a quanto detto, ad ogni estrazione 
	aggiungiamo una pallina nera nell'urna. Allora a) e c) restano processi di Markov, ma non sono 
	pi\`u omogenei
\end{exmp}

\begin{note} Se un processo \`e di Markov omogeneo per conoscerlo interamente bastano le probabilit\`a della prima transizione $\p{X_{1}=j\ |\ X_{0}=i}$ (oltre alla marginale $X_0$).
\end{note}

\begin{defn}
	\textbf{Matrice di Transizione:}
		Fissato un ordinamento degli stati di un processo di Markov omogeneo a stati finito, si definisce
		Matrice di transizione la matrice $Q=(q_ij)_{ij}$ dove
		\begin{equation*}
		 Q_{ij} = Q_{i \to j} := \p{X_1 = j \mid I \cap \{X_0 = i\}}
		\end{equation*}
Questa definizione ci permette di collezionare le probabilit\`a di transizione in una singola matrice.
\end{defn}

\begin{defn}
	\textbf{Catena di Markov}:
	Un processo di Markov omogeneo $ \{X_i\}_{i=0,\hdots,n} $ a stati finiti (o discreti) \`e  detto \textbf{Catena di Markov}.
\end{defn}
	Si possono visualizzare le Catene di Markov, data una matrice di transizione, con un grafo orientato analogo agli automi a stati finiti.
	Ad ogni stato $ i \in E $ facciamo corrispondere un nodo, e ad ogni probabilit\`a di transizione 
	$ Q_{i \to j} $ strettamente positiva facciamo corrispondere un arco $ (i, j) $. Non si disegnano gli archi delle probabilit\`a di transizione nulle. La rappresentazione con i grafi non indica nulla sulle leggi marginali della catena.

\begin{exmp}
Scriviamo la matrice di transizione per l'esempio \ref{ex-fond} nei casi a) e c).
Se associamo al rosso lo stato 1 e al blu lo stato 2
abbiamo
\begin{eqnarray*}
	\text{a): }Q = \begin{pmatrix}
	10/13 & 3/13 \\
	10/13 & 3/13
	\end{pmatrix}
		&&
	\text{c): }Q = \begin{pmatrix}
	9/12 & 3/12 \\
	10/12 & 2/12
	\end{pmatrix}
\end{eqnarray*}
Si noti che la somma delle righe della matrice deve dare 1 (rappresenta una densit\`a di probabilit\`a).
\end{exmp}

\begin{exmp}
Scriviamo per l'esempio \ref{ex-fond} nel caso c), il grafo associato alla catena di Markov
	\begin{figure}[H]
		\centering
		\caption{Catena di Markov}
		\begin{tikzpicture}[->,>=stealth',shorten >=1pt,auto,node distance=3.5cm]
		
		\node[state] (A)                    {Rossa};
		\node[state] (B) [right of=A] 	   	{Blu};
		
		\path 	(A)		edge [bend left]  	node {3/12} 		(B)
						edge [loop left] 	node {9/12} 		(A)
		(B)				edge [bend left]  	node {10/12} 		(A)
						edge [loop right] 	node {2/12} 		(B);

		\end{tikzpicture}
	\end{figure}
Si noti che in questo caso la somma dei valori {\em uscenti} da un nodo deve dare 1.
\end{exmp}



\begin{exmp}
	All'interno di una CPU abbiamo due stati, \textbf{busy} (nodo 1) e \textbf{free} (nodo 2).

	\begin{equation*}
	Q = \begin{pmatrix}
	0,3 & 0,7 \\
	0,2 & 0,8
	\end{pmatrix}
	\end{equation*}
	
	\begin{figure}[H]
		\centering
		\caption{Catena di Markov}
		\begin{tikzpicture}[->,>=stealth',shorten >=1pt,auto,node distance=3.5cm]
		
		\node[state] (A)                    {Busy};
		\node[state] (B) [right of=A] 	   	{Free};
		
		\path 	(A)		edge [bend left]  	node {0.7} 		(B)
						edge [loop left] 	node {0.3} 		(A)
				(B)		edge [bend left]  	node {0.2} 		(A)
						edge [loop right] 	node {0.8} 		(B);
				
		\end{tikzpicture}
	\end{figure}
\end{exmp}


\section{Calcolo Algebrico su catene di Markov}

\begin{prop}\textbf{Calcolo del Marginale di una Catena di Markov}
Sia data la distribuzione di probabilit\`a di trovarsi in uno stato iniziale $ \p{X_0 = j}$ $\forall j $. 
Definiamo il vettore riga $ v = (\p{X_0 = j})_j $. Sia $ Q=(q_{ji})_{ji} $ la matrice di transizione associata 
ad una catena di Markov, ovvero $ q_{ji} =q_{j \to i}= \p{X_1 = i \mid X_0 = j} $. 

Allora possiamo calcolare la legge marginale  $ \p{X_k = i} $ come:
 $$ \p{X_k = i} = (v \cdot Q^k)_i = (v \cdot  Q \cdot Q \cdot \hdots \cdot Q)_i.$$
 \end{prop}
 \begin{proof}
Dimostriamo  per induzione. Per $k=1$ abbiamo
\begin{equation*}
	\begin{aligned}
	\p{X_1 = i} =  \sum_j (\p{X_1 = i \mid X_0 = j}) \p{ X_0 = j}= \\
	 = \sum_j q_{j \to i} v_j = \sum_j v_j q_{j \to i} = (v \cdot Q)_i.
	\end{aligned}
\end{equation*}
Nel passo induttivo, supponiamo di conoscere la $k$-esima marginale $(\p{X_1 = j} )_j$, che indico 
per comodit\`a con il vettore $v^k=(\p{X_1 = j} )_j$, e che, per ipotesi induttiva valga $v_k=v\cdot Q^k$.
Allora per la $(k+1)$-esima marginale abbiamo
\begin{equation*}
	\begin{aligned}
	\p{X_{k+1} = i} =  \sum_j (\p{X_{k+1} = i \mid X_{k}= j}) \p{ X_{k} = j}= \\
	 = \sum_j q_{j \to i} v^k_j = \sum_j v^k_j q_{j \to i} = (v^k \cdot Q)_i=(v\cdot Q^k \cdot Q)_i=
	  (v\cdot Q^{k+1} )_i
	\end{aligned}
\end{equation*}

\end{proof} 

A volte, conosciamo con certezza lo stato iniziale di un sistema, ad esempio sappiamo con sicurezza 
che $ X_0 = j$. In quel caso il vettore iniziale \`e dato da $v =  e_j =\{0, 0, 0, \hdots, 0, 1, \hdots, 0, 0,\} $  
dove l'unico valore 1 \`e  in posizione $j$-esima). In quel caso per il calcolo della $k$-esima marginale si ha
$
\p{X_k = j \mid X_0 = i}=(e_i Q^k)_j = (Q^k)_{ij}. 
$

Allora anche l'insieme delle $ (Q^k)_{ij}$ al variare di $j$ \`e una densit\`a di probabilit\`a e quindi deve valere
$\sum_i (Q^k)_{ij}=1$ per ogni $j$.
\begin{exmp}\label{ex-inv}
	Riprendendo l'esempio \ref{ex-fond}, caso c):
	
	\begin{equation*}
	\begin{aligned}
	Q=\begin{pmatrix}
	\frac{9}{12} & \frac{3}{12} \\
	&\\
	\frac{10}{12} & \frac{2}{12}
	\end{pmatrix}
	\end{aligned}
	\end{equation*}
	
	Distribuzione di $ X_0 $: $ v = (\frac{10}{13}, \frac{3}{13}) $.
	
	Marginale $ X_1 $: $ v \cdot Q = (\frac{10}{13}, \frac{3}{13})\begin{pmatrix}
	\frac{9}{12} & \frac{3}{12} \\
	&\\
	\frac{10}{12} & \frac{2}{12}
	\end{pmatrix} $ = $ (\frac{120}{13 \cdot 12}, \frac{36}{13 \cdot 12}) = \left( \frac{10}{13}, \frac{3}{13} \right)$
	
	Si ha che $ \p{X_{10} = B} = (v \cdot Q^{10})_2= ((v \cdot Q) Q^9)_2 = (v \cdot Q^9)_2 =
	\dots=(v)_2 ={\dfrac{3}{13}} $
	
\end{exmp}

\paragraph{Calcolo dei valori attesi}

Data $ (X)_k $una Catena di Markov. $ Q $ matrice di transizione, $ f : \R \to \R$ funzione reale, voglio calcolare $ \E{f(X_k) \mid X_0 = i} $ (ovvero so che all'istante $X_0$ il sistema si trova nello stato $i$).

Prendiamo ad esempio il caso $ k = 1 $

\begin{equation*}
	\begin{aligned}
	\E{f(X_1)} \mid X_0 = i) = \sum_j f(j) \cdot \p{X_1 = j \mid X_0 = i} \\
	= \sum_j f(j) \cdot q_{ij} = \sum_j \cdot q_{ij}  f(j)= (Q \cdot f)_i \\
	\text{dove }\vec{f} = (f(j))_j
	\end{aligned}
\end{equation*}

\begin{prop}
In generale si ha che 
$$\E{f(X_k) \mid X_0 = i} = (Q^k\cdot \vec{f})_i $$

Se non conosco con certezza lo stato iniziale allora indichiamo, come prima, $ (P(X_0) =i)_i = v) $
e vale
\begin{equation*}
\E{f(X_k \mid X_0)} =v\cdot Q^k \cdot \vec{f}
\end{equation*}
(con $(X_k \mid X_0) $ si intende la $k$-esima marginale conoscendo la distribuzione di probabilit\`a iniziale, 
mentre con $(X_k \mid X_0=i) $ si intende la $k$-esima marginale sapendo che lo stato 
iniziale vale con certezza $i$
\end{prop}
La dimostrazione \`e simile alla dimostrazione precedente, e quindi la omettiamo.

D'ora in poi indicheremo semplicemente con $f$ il vettore $\vec{f}$, per alleggerire la notazione.


\begin{exrc} 
Prendiamo ancora l'esempio \ref{ex-fond}, caso c), e scommettiamo nel seguente modo:
se esce una pallina rossa perdo 1 euro. Se esce la blu guadagno 5 euro. Diciamo che scommettiamo sulla 
seconda estrazione. Vediamo se conviene giocare nel caso in cui sappiamo che alla prima estrazione \`e 
uscita una pallina rossa, e nel caso in cui non conosciamo la prima estrazione. 
Dobbiamo quindi calcolare, rispettivamente  $\E{f(X_1) \mid X_0 = R}$ e  $\E{f(X_1) \mid X_0 }$
Il vettore $f$ sar\`a $f=(-1,5)$, calcoliamo intanto 
	\begin{equation*}
		Q\cdot {f} = \begin{pmatrix}
	\dfrac{9}{12} & \dfrac{3}{12} \\
	&\\
	\dfrac{10}{12} & \dfrac{2}{12}
	\end{pmatrix} \begin{pmatrix}
	-1 \\ 5
	\end{pmatrix} = \begin{pmatrix}
	1/2 \\ 0
	\end{pmatrix} 	
	\end{equation*}
Quindi nel caso in cui $X_0=R$ il vettore iniziale \`e $v=(1,0)$, quindi 
  $\E{f(X_1) \mid X_0 = R}=(1,0)\cdot Q\cdot {f} =1/2$, mentre se non si conosce il risultato della prima estrazione il vettore iniziale \`e $v=(10/13,3/13)$, quindi abbiamo 
   $\E{f(X_1) \mid X_0 = R}=(10/13,3/13)\cdot Q\cdot {f} =10/26$.
  \end{exrc}


\paragraph{Catene stazionarie e stati di equilbrio}

\`E importante lo studio degli equilibri di un sistema, ovvero degli stati per cui il sistema non varia nel tempo 
(e, in certe condizioni, a cui il sistema tende nel tempo anche se si parte da uno stato che non \`e di equilibrio)
Per le catene di Markov la ricerca di questi stati non \`e difficile. 

\begin{defn}
	\textbf{Distribuzione invariante}
	Sia data una catena di Markov omogenea, con  matrice di transizione $ Q $. 
	La \textbf{distribuzione invariante} 
	per tale catena \`e  un vettore $ {\mu} = (\mu_i)_i $ tale che 	
	\begin{equation*}
	\begin{cases}
	\mu_i \geq 0 \\
	\sum_i \mu_i = 1 \\
	\vec{\mu} \cdot Q = \vec{\mu} \\
	\end{cases}
	\end{equation*}
	\end{defn}
\begin{note}
Si osservi che $ {\mu}^\tau $ \`e  l'autovettore di $ Q^\tau$riferito all'autovalore 1,  quindi se una
distribuzione invariante per $Q$ esiste, basta risolvere $(Q^\tau-Id)\mu^\tau = 0$
\end{note}

\begin{defn}
	\textbf{Catena Stazionaria}
	
	Una Catena di Markov $(X_k)_k$ \`e  una \textbf{catena stazionaria} se \textbf{tutte} le sue marginali sono uguali, ovvero se $\p{X_k=i}=p{X_0=i}$ per ogni $i,k$.
	\end{defn}
	
 Se la catena  $(X_k)_k$ \`e  stazionaria, se prendiamo il vettore della prima marginale 
	 $\mu=(p{X_0=i})_i$ abbiamo che $\mu\cdot Q^k=\mu$ per ogni $k$. Ma allora 
	 $\mu=(p{X_0=i})_i$  \`e una distribuzione invariante per $Q$. Viceversa, se abbiamo una distribuzione
	 invariante $\mu$ per una certa catena di Markov con matrice di transizione $Q$, allora tale 
	 catena \`e stazionaria, infatti
	 $$
	 \mu\cdot Q^k=(\mu\cdot Q)\cdot Q^{k-1}=\mu\cdot Q^{k-1}=\dots=\mu\cdot Q=\mu.
 $$
Quindi 
$$ (X_k)_k \text {\`e una catena di Markov stazionaria} \Leftrightarrow (p{X_0=i})_i
 \text{ \`e una sua distribuzione invariante)}
$$	 

\begin{exmp} 
Abbiamo visto nell'esempio \ref{ex-inv} che il vettore $ \left( \frac{10}{13}, \frac{3}{13} \right)$ \`e una
distribuzione invariante per la catena di Markov dell'esempio \ref{ex-fond}, caso c). Siamo stati fortunati 
o questo processo \`e una catena invariante indipendentemente dal numero di biglie blu e rosse?

Supponiamo di avere quindi $ N = B + R$ biglie totali, di cui $R $ Rosse e $ B $ Blu. Il primo nodo sono le biglie rosse ed 
il secondo quelle blu, al solito.
Le probabilit\`a di transizione, e la matrice $Q$ saranno quindi 	
	\begin{equation*}
	\begin{aligned}
	q_{1,1} = \p{X_1 = R \mid X_0 = R} = \dfrac{R-1}{N-1} \\
	q_{1,2} = \p{X_1 = B \mid X_0 = R} = \dfrac{B}{N-1} \\
	q_{2,1} = \p{X_1 = R \mid X_0 = B} = \dfrac{R}{N-1} \\
	q_{2,2} = \p{X_1 = B \mid X_0 = B} = \dfrac{B-1}{N-1} \\
	Q = \begin{pmatrix}
	\dfrac{R-1}{N-1} & \dfrac{B}{N-1} \\
	&\\
	\dfrac{R}{N-1} & \dfrac{B-1}{N-1}
	\end{pmatrix}
	\end{aligned}
	\end{equation*}
	
	Verifichiamo che lo stato iniziale $ \mu = \left(\dfrac{R}{N}, \dfrac{B}{N}\right) $ \`e  una distribuzione invariante.
	
	\begin{equation*}
	\begin{aligned}
	\mu Q = \left(\dfrac{R}{N}, \dfrac{B}{N}\right)\begin{pmatrix}
	\dfrac{R-1}{N-1} & \dfrac{B}{N-1} \\
	&\\
	\dfrac{R}{N-1} & \dfrac{B-1}{N-1}
	\end{pmatrix} \\
	= \left(\dfrac{R(R+B-1)}{N(N-1)} , \dfrac{B(R+B-1)}{N(N-1)}\right) = \left(\dfrac{R}{N}, \dfrac{B}{N}\right)
	\end{aligned}
	\end{equation*}
\end{exmp}



\begin{defn}
	\textbf{Matrice di Transizione regolare}
	Una matrice di transizione $ Q $ si dice regolare se per qualche $k$  si ha che $(Q^k)_{ij} > 0 $.
	 \end{defn}
Se $ Q $ \`e  regolare e $ v $ \`e  uno stato iniziale qualsiasi allora $ v\cdot Q^k $ tende alla distribuzione invariante per la catena di 
Markov. Quindi, in questo caso, da qualsiasi stato iniziale si converge verso una distribuzione che possiamo calcolare a priori. 
Quindi il comportamento di questo sistema diventa prevedibile su tempi lunghi. 

\begin{exmp}
	All'interno di una CPU abbiamo due stati, \textbf{busy} (nodo 1) e \textbf{free} (nodo 2).
	
	\begin{equation*}
	Q = \begin{pmatrix}
	0,3 & 0,7 \\
	0,2 & 0,8
	\end{pmatrix}
	\end{equation*}
	
	Cerco $ \mu $ distribuzione invariante. Sappiamo che $ \mu $ \`e  autovettore di autovalore 1:
	
	\begin{equation*}
	\begin{aligned}
	\mu Q = \mu \iff Q^\tau \mu^\tau  = \mu^\tau \iff (Q^\tau -I)\mu^\tau = 0 \\
	Q^\tau - I = \begin{pmatrix}
	0.3 & 0.2 \\
	0.7 & 0.8
	\end{pmatrix} - \begin{pmatrix}
	1 & 0 \\ 0 & 1
	\end{pmatrix} = \begin{pmatrix}
	-0.7 & 0.2 \\
	0.7 & -0.2
	\end{pmatrix} \\
	\begin{pmatrix}
	-0.7 & 0.2 \\
	0.7 & -0.2
	\end{pmatrix} \begin{pmatrix}
	\mu_1 \\ \mu_2
	\end{pmatrix} = 0 
	\implies \begin{cases}
	-0.7\mu_1 + 0.2 \mu_2 = 0 \\
	0.7\mu_1 - 0.2 \mu_2 = 0 \\
	\mu_1 + \mu_2 = 1\\
	\mu_1 \geq 0; \mu_2 \geq 0	

	\end{cases}
	\end{aligned}
	\end{equation*}
	
Risolvendo il sistema si ottiene
$$
	\mu=\left( \dfrac{2}{9},\dfrac{7}{9}\right)
$$	
\end{exmp}

Si noti che la matrice di transizione dell'esercizio precedente \`e regolare. 
Quindi se partiamo da un qualsiasi stato, per tempi molto lunghi lo stato del sistema si avviciner\`a alla distribuzione invariante.
In altre parole, dopo abbastanza tempo la probabilit\`a di trovare la CPU occupata sar\`a circa di $2/9$, mentre 
la probabilit\`a di trovarla  occupata sar\`a circa di $7/9$.

\section{Esercizi}

\begin{exrc}
	\textbf{Passeggiata Aleatoria}
	Mi muovo nell'asse X casualmente partendo da 0. Al minuto $ k $ lancio una moneta. Se esce testa mi muovo a destra, se esce croce mi muovo a sinistra. Voglio ottenere la posizione al minuto k. Disegnare la catena di Markov come grafo.
	
	\begin{equation*}
	\begin{aligned}
	Y_k = \text{Bern}\left(\dfrac{1}{2}\right) \text{ (lancio della moneta)}\\
	Y_k \in \{-1, +1\} \\
	X_k = \text{posizione} \\
	\begin{cases}
	X_0 = 0 \\
	X_{k+1} = X_k + Y_k
	\end{cases}
	\end{aligned}
	\end{equation*}
	
	\begin{figure}[H]
		\centering
		\caption{Catena di Markov della passeggiata aleatoria}
		\begin{tikzpicture}[->,>=stealth',shorten >=1pt,auto,node distance=2cm]
		
		\node[state] (A)                    {0};
		\node[state] (B) [right of=A] 	   	{1};
		\node[state] (C) [right of=B] 	   	{2};
		\node[state] (D) [right of=C] 	   	{$ \hdots $};
		\node[state] (E) [left of=A] 	   	{-1};
		\node[state] (F) [left of=E] 	   	{-2};
		\node[state] (G) [left of=F] 	   	{$ \hdots $};		
		;
		
		\path 	
		(A)		edge [bend left]  	node {1/2} 		(B)
				edge [bend left]  	node {1/2} 		(E)
		(B)		edge [bend left]  	node {1/2} 		(C)
				edge [bend left]  	node {1/2} 		(A)
		(C)     edge [bend left]  	node {1/2} 		(D)
				edge [bend left]  	node {1/2} 		(B)
		(E)		edge [bend left]	node {1/2}		(F)
				edge [bend left]  	node {1/2} 		(A)
		(F)		edge [bend left]  	node {1/2} 		(G)
				edge [bend left]  	node {1/2} 		(E);
		\end{tikzpicture}
	\end{figure}
\end{exrc}


\begin{exrc}
	Invece di prendere una passeggiata aleatoria, immaginiamo di voler 
	schematizzare una passeggiata di un ubriaco che si muove senza meta, quindi in modo aleatorio, ma 
	che \`e  restio a cambiare direzione.
	Quindi se all'istante $ k $ \`e  andato a sinistra, all'istante $ k+1 $ la direzione sinistra sar\`a  pi\`u  probabile della destra. La sua posizione \`e  una catena di Markov? 
	
	No, perch\'e  dipende dalla posizione all'istante precedente e dalla direzione. La scelta di andare a destra o sinistra, invece, pu\`o essere rappresentata da una Catena di Markov, perch\'e  dipende soltanto dall'ultima scelta fatta.
\end{exrc}



\begin{exrc}
	
	Vogliamo simulare un essere vivente elementare con un automa. I suoi stati sono (1) \textbf{relax}, (2) \textbf{vigile}, (3) \textbf{fuga}, (4) \textbf{attacca}. Sappiamo che in presenza di qualche segnale esterno dallo stato di relax l'animale passa
	allo stato vigile, da qui pu\`o tornare allo stato di relax (con probabilit\`a $1/2$), oppure decidere di attaccare 
	(con probabilit\`a $1/5$) o fuggire (con probabilit\`a $3/10$). Scrivere il grafo e la matrice di transizione legata a questo processo.
	
Questo modello non \`e del tutto soddisfacente, perch\`e, ad esempio, il nostro automa uscir\`a dallo stato di relax a intervalli prestabiliti. 
In natura invece il cambiamento di stato di un animale dipende dagli stimoli esterni, che non sono ad intervalli regolari. 
Una maniera di introdurre quindi un tempo in una catena di Markov che non coincida necessariamente con gli intervalli a cui 
si studia il processo, si pu\`o introdurre per ogni stato $i$ una probabilit\`a $p_i$ che lo stato vada in se stesso. Provare a scrivere 
la matrice di transizione per l'automa cos\`\i\ modificato.
	
	\begin{figure}[H]
		\centering
		\caption{Catena di Markov dell'automa cellulare}
		\begin{tikzpicture}[->,>=stealth',shorten >=1pt,auto,node distance=3.5cm]
		
		\node[state] 	(A)                    {R};
		\node[state]         	(B) [right of=A] 	   {V};
		\node[state]         	(C) [below right of=B] 	   {F};
		\node[state]         	(D) [above of=B] 	   {A};
		
		\path 	(A)		edge [bend left]  	node {1} 		(B)
		(B)
		edge [bend left] 	node {$\dfrac{1}{2}$} 		(A)
		edge [bend left]  	node {$\dfrac{1}{5}$} 		(D)
		edge [bend left]  	node {$\dfrac{3}{10}$} 		(C)
		(C)		edge [bend left]  	node {1} 		(B)
		(D)		edge [bend left]  	node {1} 		(B);
		\end{tikzpicture}
	\end{figure}
	
	\begin{equation*}
	\begin{aligned}
	Q = \begin{pmatrix}
	0 & 1 & 0 & 0 \\
	\dfrac{1}{2} & 0 & \dfrac{3}{10} & \dfrac{1}{5} \\
	0 & 1 & 0 & 0 \\
	0 & 1 & 0 & 0 
	\end{pmatrix}
	\end{aligned} 
	\end{equation*}
	
	\begin{figure}[H]
		\centering
		\caption{Catena di Markov dell'automa cellulare con tempo}
		\begin{tikzpicture}[->,>=stealth',shorten >=1pt,auto,node distance=3.5cm]
		
		\node[state] 	(A)                    {R};
		\node[state]      (B) [right of=A] 	   {V};
		\node[state]      (C) [below right of=B] 	   {F};
		\node[state]      (D) [above of=B] 	   {A};
		
		\path 	(A) edge [bend left]  	node {$1-p_1$} (B) 
		edge [loop left] node {$p_1$} (A)
		(B) edge [loop left] node {$p_2$} 
		(C) edge [bend left] node {$ \dfrac{1}{2} (1-p_2) $} 		
		(A) edge [bend left]  	node {$\dfrac{1}{5}(1-p_2)$} 		
		(D)
		edge [bend left]  	node {$\dfrac{3}{10}(1-p_2)$} 		(C)
		(C)		edge [bend left]  	node {$1-p_3$} 		(B)
		edge [loop right] node {$p_3$} (C)
		(D)		edge [bend left]  	node {$1-p_4$} 		(B)
		 edge [loop right] node {$p_4$} (D);
		\end{tikzpicture}
	\end{figure}
\end{exrc}


\begin{exrc}


Abbiamo una CPU con 3 stati: (1) \textbf{Off}, (2) \textbf{Stand By}, (3) \textbf{Busy}

\begin{figure}[H]
	\centering
	\caption{Catena di Markov della CPU a 3 stati}
	\begin{tikzpicture}[->,>=stealth',shorten >=1pt,auto,node distance=3.5cm]
	
	\node[state] 			(A)                    {O};
	\node[state]         	(B) [right of=A] 	   {S};
	\node[state]         	(C) [right of=B] 	   {B};
	
	\path 	(A)		edge [bend left]  	node {?} 		(B)
	edge [loop below] node {?} (A)
	(B)
	edge [loop below] node {0.4} (B)
	edge [bend left] 	node {?} 		(A)
	edge [bend left]  	node {0.2} 		(C)
	(C)		edge [bend left]  	node {0.6} 		(B)
	edge [loop below] node {0.4} (C);
	\end{tikzpicture}
\end{figure}


\begin{enumerate}
	\item Completare gli archi del grafo. Si possono completare sapendo che la somma degli archi uscenti da un nodo dev'essere 1.
	La matrice $Q$ sar\`a 
	$$Q = \begin{pmatrix}
		0.8 & 0.2 & 0 \\
		0.2 & 0.4 & 0.4 \\
		0 & 0.6 & 0.4
		\end{pmatrix} $$
	\item Calcolare $ \p{X_1 = O \mid X_0 = O} $ e $ \p{X_2 = O \mid X_0 = O} $. Abbiamo
	$\p{X_1 = O \mid X_0 = O}= q_{00} = 0.8 $. Calcoliamo ora $\p{X_2 = O \mid X_0 = O}$. Il vettore corrispondente 
	allo stato iniziale $X_0 = O$ \`e $v=(1,0,0)$, quindi 
	\begin{equation*}
				\p{X_2 = O \mid X_0=O} = (v \cdot Q \cdot Q)_1 = (0.68, \cdots, \cdots)_1=0.68
					\end{equation*}

	\item Supponiamo di avere una funzione costo tale che $c(O)=0$,  $c(S)=5$ e $c(B)=10$. Calcolare $ \E{c(X_k) \mid X_0 = O} $ per $ k = 1,2 $.
	Per $ k = 1 $ possiamo procedere in modo elementare  
	\begin{equation*}
		\begin{aligned}
		\E{c(X_1) \mid X_0 = O} = 0 \cdot \p{X_1 = O \mid X_0 = O } \\
		+ 5 \cdot \p{X_1 = S \mid X_0 = O} + 10 \cdot \p{X_1 = B \mid X_0 = O} \\
		= 5 \cdot \p{X_1 = S \mid X_0 = O} = 5 \cdot 0.2 = 1 
		\end{aligned}
	\end{equation*}
	Per $ k = 2 $ invece conviene usare la formula vista precedentemente. Il vettore associato al costo \`e $f = (0, 5, 10)$ quindi
	\begin{equation*}
			\E{c(X_2) \mid X_0 = O} = (Q^2 \cdot f)_1 = v\cdot Q^2\cdot f =  2
	\end{equation*}
	\item Calcolare la varianza $\var{c(X_1) \mid X_0 = 0}$. 
	Abbiamo
	\begin{equation*}
		\begin{aligned}
				c^2=(0, 25, 100) \\
		\E{c^2(X_1 \mid X_0 = 0}) = 0 \cdot \p{X_1 = O \mid X_0 = O } \\
		+ 25 \cdot \p{X_1 = S \mid X_0 = O } + 100 \cdot \p{X_1 = B \mid X_0 = O } \\ 
		= 25 \cdot 0.2 = 5
		\end{aligned}
	\end{equation*}
	quindi $$\var{c(X_1) \mid X_0 = 0} = 
		\E{c^2(X_1 \mid X_0 = 0)} - (\E{c(X_1) \mid X_0 = 0})^2 =5-1=4$$

	\item Calcolare $\mu$ distribuzione invariante e $\E{c(X_1) \mid \mu}$
\begin{equation*}
	\begin{aligned}
	(\text{Calcoliamo } \mu )\\ \\
	(Q^\tau - I) \mu^\tau = 0 \\
	\mu = (\mu_1, \mu_2, \mu_3) \\
	(Q^\tau - I)\mu^\tau = \begin{pmatrix}
	-0.2 & 0.2 & 0 \\ 
	0.2 & -0.6 & 0.6 \\
	0 & 0.4 & -0.6
	\end{pmatrix} \mu^\tau = 0 \\
	\implies \begin{cases}
	-0.2 \mu_1 + 0.2 \mu_2 = 0 \\
	0.2 \mu_1 - 0.6 \mu_2 + 0.6 \mu_3 = 0 \\
	 0.4 \mu_2 - 0.6 \mu_3 = 0 \\
	 \mu_1 + \mu_2 + \mu_3 = 1
	\end{cases} 
	\implies \begin{cases}
	\mu_1 = \mu_2 \\
	\mu_3 = \dfrac{2}{3} \mu_2 \\
 	\mu_1 + \mu_2 + \mu_3 = 1
	\end{cases} \\
	\implies \mu = \left(\dfrac{3}{8}, \dfrac{3}{8}, \dfrac{1}{4}\right) \\ \\
	(\text{Calcoliamo } \E{c(X_1) \mid \mu}) \\ \\
	\E{c(X_1) \mid \mu} = \mu \cdot Q \cdot f =4.375
		\end{aligned} 
\end{equation*}
\end{enumerate}
\end{exrc}

\begin{exrc}
	Dato il grafo di una Catena di Markov
	
	\FloatBarrier 
	\begin{figure}[H]
		\centering
		\caption{Catena di Markov}
		\begin{tikzpicture}[->,>=stealth',shorten >=1pt,auto,node distance=3.5cm]
		
		\node[state] 			(A)                    {1};
		\node[state]         	(B) [below left of=A] 	   {2};
		\node[state]         	(C) [below right of=A] 	   {3};
		
		\path 	(A)		edge []  	node {0.5} 		(B)
		edge [loop above] node {0.5} (A)
		(B)
		edge []  	node {1} 		(C)
		(C) edge [] node {1} (A);
		\end{tikzpicture}
	\end{figure}

	\begin{enumerate}
		\item Trovare Q matrice di transizione 
		\begin{equation*}
		Q = \begin{pmatrix}
		0.5 & 0.5 & 0 \\
		0 & 0 & 1 \\ 
		1 & 0 & 0
		\end{pmatrix}
		\end{equation*}
		
		\item Scrivere le marginali $ X_1, X_2, X_3 $ sapendo che $ X_0 = 1 $
		
		\begin{equation*}
				\begin{aligned}
				v = (1, 0, 0) \\ 
				\p{X_1 \mid X_0 = 1} = v \cdot Q = \left(\dfrac{1}{2}, \dfrac{1}{2}, 0\right) \\
				\p{X_2 \mid X_0 = 1} = v \cdot Q^2 = \left(\dfrac{1}{2}, \dfrac{1}{2}, 0\right) Q = \left( \dfrac{1}{4}, \dfrac{1}{4}, \dfrac{1}{2} \right) \\
				\p{X_3 \mid X_0 = 1} = v \cdot Q^3 = \left(\dfrac{1}{2}, \dfrac{1}{2}, 0\right) Q^2 = \left(\dfrac{5}{8}, \dfrac{1}{8}, \dfrac{1}{4}\right)\\
				\end{aligned}
		\end{equation*}
	
		\item Calcolare la distribuzione invariante $ \mu $ e $ \E{\mu} $ 
		\begin{equation*}
			\begin{aligned}
			(Q^T-I)\mu^\tau = \begin{pmatrix}
			-0.5 & 0 & 1 \\
			0.5 & -1 & 0 \\
			0 & 1 & -1
			\end{pmatrix} \begin{pmatrix}
			\mu_1 \\ \mu_2 \\ \mu_3
			\end{pmatrix} \\
			\implies \begin{cases}
			- \mu_1 + 2\mu_3 = 0 \\ 
			\mu_1 - 2\mu_2 = 0 \\
			2\mu_2 - 2\mu_3 = 0
			\end{cases} \implies \begin{cases}
			\mu_1 = 2\mu_3 \\
			\mu_1 = 2\mu_2 \\
			\mu_1 + \mu_2 + \mu_3 = 1
			\end{cases} 
			\implies \mu = \left(\dfrac{1}{2}, \dfrac{1}{4}, \dfrac{1}{4}\right) \\ 
			\iff \begin{cases}
			\p{X_0 = 1} = \frac{1}{2} \\
			\p{X_0 = 2} = \frac{1}{4} \\
			\p{X_0 = 3} = \frac{1}{4} \\	
			\end{cases} \\
			\E{\mu} = 1 \cdot \p{X_0 = 1}  + 2 \cdot \p{X_0 = 2}  + 3 \cdot \p{X_0 =3 } =  1 \cdot \dfrac{1}{2} + 2 \cdot \dfrac{1}{4} + 3 \cdot \dfrac{1}{4} = \dfrac{7}{4}
			\end{aligned} 
		\end{equation*}
		
		\item Se la catena \`e  stazionaria calcolare $ \p{X_1 = 1 \mid X_3 = 1} $. Utilizziamo la formula di Bayes. 
		
		\begin{equation*}
			\begin{aligned}
			\p{X_1 = 1 \mid X_3 = 1} = \p{X_3 = 1 \mid X_1 = 1} \cdot \dfrac{\p{X_1 = 1}}{\p{X_3 = 1}} \\
			\text{La catena \`e  stazionaria: } \p{X_3 = 1} = \p{X_1 = 1} \\
			\implies \p{X_1 = 1 \mid X_3 = 1} = \p{X_3 = 1 \mid X_1 = 1} \\ 
			= \p{X_2 = 1 \mid X_0 = 1} = (1, 0, 0) Q^2 = \dfrac{1}{2}
			\end{aligned}
		\end{equation*}
	\end{enumerate}
\end{exrc}







\backmatter
% bibliography, glossary and index would go here.

\end{document}