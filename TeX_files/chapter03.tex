\chapter{Variabili Aleatorie}

\begin{figure}[H]
	\centering
	\caption{Tipi di variabili casuali}
	\includegraphics[]{varrandom}
\end{figure}

\section{Variabili Aleatorie Discrete}

Una variabile aleatoria (casuale o stocastica) discreta è una variabile che può assumere diversi valori in dipendenza da qualche fenomeno casuale.
Il risultato del lancio di un dado, ad esempio, è una variabile aleatoria discreta.

Prendiamo uno \textbf{spazio probabilizzabile} $ (\Omega, F) $ e una variabile aleatoria discreta $ X : \Omega \to \R $, che è una funzione continua non surgettiva. I valori di $ X $ devono essere un sottoinsieme finito di $ \R $ ovvero $ \{a_1, \dots, a_k\} $.
Vogliamo anche che $ \forall j \in [1, k] $ sia vero $ X^{-1}(a_j) = \{ \omega \in \Omega, X(\omega) = a_j\} \in F $. Utilizziamo la funzione inversa per ottenere gli elementi di $ \Omega $ su cui possiamo definire la probabilità.

Con la probabilità di tutti gli eventi definisco la densità di probabilità. Nello spazio probabilizzato $ (\Omega, F, \mathcal{P}) $ la probabilità che la variabile aleatoria assuma il valore $ a_j $ sarà $ p_j = \p{X = a_j} = \p{X^{-1}(a_j)} $, che viene detta densità di probabilità. Varranno quindi le seguenti proprietà

\begin{enumerate}
	\item $ \forall j \ldotp X^{-1} (a_j) $ sono tutti insiemi disgiunti.
	\item Essi coprono tutto $ \Omega $
\end{enumerate}

Vale che $ \sum_{j=1}^{k} p_j = 1 $. Poiché:
\[ 1 = \p{\Omega} = \p{\bigcup_{j} X^{-1}(a_j)} = \sum_{j} \p{X^{-1}(a_j)}\] 

Sia $ X : \Omega \to \{ a_1, \dots, a_k \} $ una variabile aleatoria, e sia la densità di probabilità $ p_j \geq 0 $ e anche  $ \sum_{j=1}^{k} p_j = 1 $, allora $ \p{X = a_j} = p_j $.

Preso uno spazio probabilizzabile $ (\Omega, F) $ e una variabile aleatoria $ X : \Omega \to \{ a_1, \dots, a_k \} $, supponendo che i numeri $ a_j $ siano ordinati, sia $ p_j $ la densità di probabilità, come posso ricostruire, ad esempio $ \p{x \leq a_3} $?

\[ \p{X \leq a_3} = \p{(X=a_1) \cup (X=a_2) \cup (X=a_3)} = \p{X = a_1} + \p{X = a_2} + \p{X = a_3}  \]

\paragraph{Esempio}

Voglio contare quanti 6 escono in 10 lanci di dadi.

Sia $ \Omega = \{ (1, 2, 3, 4, 5, 6)\}^{10} $, ovvero tutte le possibili parole di 10 elementi composte dai numeri da 1 a 6. Ad ogni lancio, ho $ \frac{1}{6} $ di probabilità di ottenere 6 e $ \frac{5}{6} $ di ottenere gli altri numeri. Definiamo la variabile aleatoria $ X : \Omega \to \{ 0, 1, 2, \dots, 9, 10 \} $ come il conteggio dei risultati dei lanci dove ottengo 6. Qual'è la probabilità di ottenere 3 lanci dove ho fatto 6?

\[ \p{X = 3} = \binom{10}{3} \left(\dfrac{1}{6}\right)^3 \left(\dfrac{5}{6}\right)^7 \]

\subsection{Legge di Bernoulli}
Faccio un esperimento, il risultato positivo ha probabilità $ p $, mentre il risultato negativo ha probabilità $ 1 - p $

Sia lo spazio $ \Omega = \begin{cases}
\text{successo} \to 1 \\
\text{insuccesso} \to 0
\end{cases} $

Una variabile aleatoria Bernoulliana è definita come $ X : \Omega \to \begin{cases}
p_1 = p \\ p_0 = 1 - p
\end{cases} $

\subsection{Legge Binomiale} 
Sia $ k $ il conteggio dei successi di $ n $ esperimenti, abbiamo quindi che $ B(n,p) = X_i\{(0,1)\}^n \to \{0, \dots, n\} $. Abbiamo che la densità di probabilità Binomiale $ p_k = \p{X=k} = \binom{n}{k} p^k(1-p)^{n-k} $.

$ p_k $ è una densità? Sappiamo che $ p_k \geq 0 $ e sappiamo che $ 1 = \sum_{k} p_k $, con il binomio di Newton possiamo dimostrare che $ (a+b)^n = \sum_{k=0}^{n} \binom{n}{k} a^k b^{n-k} $. Proseguendo, abbiamo che 

\[ \sum_{k} p_k = \sum_{k=0}^{n} \binom{n}{k} p^k (1-p)^{n-k} = (p+(1-p)^n) = 1^n = 1 \]

\subsection{Somma di Variabili Aleatorie}
Lancio due dadi, uno rosso e uno nero, avremo quindi $ \Omega = \{R, N\} = \{(1,6)\}^2 $. Definisco due variabili aleatorie, $ X $ per il dado rosso dove $ X : (R, N) \to R $ e la variabile $ Y : (R, N) \to N $. La densità per $ X $ sarà $ p_j = \frac{1}{6} \forall j $ mentre la densità per $ Y $ sarà $ q_j = \frac{1}{6} \forall j $
Avremo che $ Z = X + Y $ conta la somma dei dadi.

\paragraph{Esercizi}
\begin{enumerate}
	\item Calcolare la densità di $ Z $
		
		In questo caso $ X, Y $ sono indipendenti, quindi avremo la densità di $ Z $ detta $ t_Z $.
		
		\[ \forall n \in [2, 12] .  \left( \p{Z = n} = \sum_{i=1}^{n-1} \p{X = i} \cdot \p{Y = n - i} \right) \]
		
		\begin{table}[H]
			\centering
			\caption{Distribuzione discreta della somma del lancio di due dadi.}
			\label{tab:distribdice1}
			\begin{tabular}{|l|l|l|l|l|l|l|l|l|l|l|l|}
				\hline
				$ X + Y $     & 2        & 3        & 4        & 5        & 6        & 7        & 8        & 9        & 10       & 11       & 12       \\ \hline
				$ p_j + q_j $ & $ 1/36 $ & $ 2/36 $ & $ 3/36 $ & $ 4/36 $ & $ 5/36 $ & $ 6/36 $ & $ 5/36 $ & $ 4/36 $ & $ 3/36 $ & $ 2/36 $ & $ 1/36 $ \\ \hline
			\end{tabular}
		\end{table}
	
	\begin{figure}[H]
		\centering
		\caption{Distribuzione della somma del lancio di due dadi}
		\includegraphics[]{binomdadi}
	\end{figure}
		
	\item Calcolare $ \p{4 \leq Z \leq 6} $
	
	$ \p{4 \leq Z \leq 6} = \p{Z = 4} + \p{Z = 5} + \p{Z = 6} = \dfrac{3}{36} + \dfrac{4}{36} + \dfrac{5}{36} = \dfrac{12}{36} = \dfrac{1}{3} $
\end{enumerate}

\subsection{Indipendenza di Variabili Aleatorie}

Due variabili aleatorie $ X_1, X_2 $ sono indipendenti se $ \forall I_1, I_2, \subseteq \R $ intervalli o semirette si ha che 

\[ \left( \p{X_1 \in I_1} \cap \p{X_2 \in I_2} \right) = \p{X_1 \in I_1} - \p{X_2 \in I_2} \]

Nell'esempio di prima $ X, Z $ e $ Y,Z $ sono dipendenti perché dati $ I_1 = [1,2], I_2 = [3,4] $ allora si ha che $ \p{X\in I_1} = \p{X=1,2} = \dfrac{1}{3} $ e si ha anche $ \p{Z \in I_2} = \p{2 = 3,4} = \dfrac{5}{36} $. Ne otteniamo che:

% TODO fix erratum

\[ \p{(X \in I_1) \cap (Z = 3,4)} = \p{X = 1,2, Z = 3,4} = \dfrac{4}{12} \]

\paragraph{Variabili Aleatorie Congiunte}
Due variabili aleatorie $ X,Y : \Omega \to \R $ discrete sono congiunte quando si può calcolare $ \p{X = m \cap Y = n} = p_{n,m} $, ovvero una densità di probabilità $ \{ p_{n,m} \}_{n,m} $ con $ (p_{n,m} \geq 0 ) \land (\sum_{n,m} p_{n,m} = 1) $.

Sapendo $ p_{n,m} $ ricavo tutti i $ \p{X=m} = p_m $ e $ \p{Y = n} = q_n $ con

\[ \p{X=m} = \p{X = m \cap \left\{ \bigcup_{n} Y=n \right\}} = \sum_{n} \p{X=m,Y=n} = \sum_n p_{n,m} = p_m \]
\[ \p{Y=n} = \sum_{m} p_{n,m} = q_n \]

Non si può ricostruire dalle due probabilità la variabile aleatoria congiunta. Ad esempio, conoscendo $ p_n,q_n $ cerco $ p_n,m  = \p{X = m \cap Y = n} = \p{X = m | Y = n} \cdot \p{Y=n} $. Se $ X,Y $ sono indipendenti allora $ P(X=m \mid Y=n) = \p{X=m} $ e vale il prodotto $ p_{m,n} = p_n \cdot p_m $.

\[ \sum_{m}\sum_{n} p_{n,m} = \sum_{m}\sum_{n} p_n \cdot q_n = \left( \sum_{m} p_m \right) \left( \sum_{n} q_m \right) = 1 \]

\paragraph{Formula di Convoluzione}

Tornando alla somma di due variabili aleatorie discrete, dati $ X,Y $ indipendenti e $ Z = X + Y $, con $ X,Y : \Omega \to \N $, dobbiamo calcolare $ \p{Z = n} $ e la densità di probabilità discreta $ \left\{Z = n\right\} $

\[ \left\{Z = n\right\} = \bigcup_{i=0}^{n} \left\{X = 1 \cap Y = n - i \right\} \]

\[ \p{z=n} = \sum_{i} \p{X = i \cap Y = n-1} = \sum_{i = n}^{n} \p{X=i} \cdot \p{Y=n-i} \]

\paragraph{Teorema: Rapporto fra Bernoulli e Binomiale}

Sommando $ n $ esperimenti di $ p $ dove conto i successi ottengo la binomiale $ B(n,p) $, quindi $ \sum_{i=1}^{n} \text{Bern}_i(p) = B(n,p) $

\paragraph{Dimostrazione per Induzione}
Caso base, per $ n = 1 \implies B(1,p) = \text{Bern}(p)$. Come passo induttivo abbiamo 
\[ B(n,p) = \sum_{i=1}^{n} \text{Bern}_i(p) \implies B(n+1,p) = \sum_{i=1}^{n+1} = \text{Bern}_i(p) \]
\[ \sum_{i=1}^{n+1} \text{Bern}_i(p) = \left( \sum_{i=1}^{n} \text{Bern}_i(p) \right) + \text{Bern}_{n+1}(p) = B)(n,p) + \text{Bern}(p) \]

Introduciamo le densità per continuare la dimostrazione
\[ \p{B(n,p) + \text{Bern}(p) = k} = \p{B(n,p) = k \cap \text{Bern}(p) = 0} + \p{B(n,p) = k-1 \cap \text{Bern}(p) = 1} \]
\[ = \binom{n}{k-1}p^{k-1}(1-p)^{n-k+1} \cdot p + \binom{n}{k}p^k(1-p)^{n-k} \cdot  (1-p) \]
\[ = \left[\binom{n}{k-1}\binom{n}{k}\right] p^k (1-p)^{n-k+1} = \binom{n+1}{k} p^k (1-p)^{n+1-k} = \p{B(n+1,p) = k}\]

\section{Variabile Geometrica}
Data una successione di esperimenti ripetuti con probabilità di successo $ 0 \leq p \leq 1 $. Lo ripeto fino ad ottenere un successo. $ \geom(p) $ conta il numero di prove necessarie. Ovvero $ \p{\geom(p) = k} = $ la probabilità di fare k esperimenti ed avere un successo dall'ultimo. La densità di probabilità sarà $ p_k = \p{\geom(p) = k} = (1-p)^{k-1} \cdot p $ con $ p_k \geq 0 $ e anche $ \sum_{k=1}^{\infty} (1-p)^{k-1} p = p \sum_{k=1}^{\infty}(1-p)^{k-1} $, che è una serie geometrica. Sarà quindi equivalente a $ p \sum_{i=0}^{n} (1-p)^i = \dfrac{p}{1-(1-p)} = 1$


Osservazione: consideriamo la serie geometrica $ \sum_{i=0}^{\infty} q^i $.
Sappiamo che $ (1-q) \sum_{i=0}^{\infty} q^i = 1 $ perché $ (1-q)(1+q+q^2+q^3\dots) $ = $( 1 - q + q - q^2 + q^2 + \dots )$, semplificando i termini rimane 1.

Una variabile geometrica \textbf{non ha memoria}, l'esperimento numero $ n $ ha la stessa probabilità degli altri esperimenti:

\[ \p{\geom(p) = n+m | \geom(p) > n} = \p{\geom(p) = m}  = \]
\[ = \dfrac{\p{\geom(p) = m + n \cap \geom(p) > n}}{\p{\geom(p) > n}} = \dfrac{\p{\geom(p) = m+n}}{\p{\geom(p) > n}}\] 
\[= \dfrac{(1-p)^{m+n-1} \cdot p}{(1-p)^n} = \p{\geom(p) = m} \]

Si può dimostrare sapendo che $ \forall m . \p{\geom(p) = n+m \cap \p{\geom} > n} = \p{\geom(p) = m+n} $

\subsection{Variabili Ipergeometriche}

Siano dati $ r $ sfere rosse, $ b $ sfere bianche, $ n $ estrazioni senza reimbussolamento, $ k = $ numero di sfere rosse estratte, $ H(b+r, r, n) $ conta il numero di sfere rosse estratte dopo n tentativi. Sappiamo che $ (0 < n \leq b + r) \land (k \leq n)  \land (k \leq r) \land (n-k \leq b) $.

Definiamo la densità di probabilità
\[ \p{H(b+r, r, n) = k} = \dfrac{\binom{r}{k} \binom{b}{n-k}}{\binom{b+r}{n}} \] 

\subsection{Esercizi}

\begin{enumerate}
	\item Dati $ r = 10, 15 = b, 7 = n $ abbiamo che $ \p{H(25,15,7) = 3} $ è $ \dfrac{\binom{15}{3}\binom{10}{7-3}}{\binom{25}{7}} $
	
	\item Ho una scatola con 12 lampadine, 4 di esse sono fulminate. Ne prendo 2. La probabilità che siano entrambe funzionanti è $ \p{H(12, 4, 2) = 0} = \dfrac{\binom{4}{0}\binom{8}{2}}{\binom{12}{2}} $
	
	\item Ho una moneta truccata. La probabilità che esca testa è $ P_t = 0.55 $ e la probabilità che esca croce è $ P_c = 0.45 $. Lancio la moneta dieci volte, qual'è la probabilità che avvenga la sequenza testa-croce per la prima volta al lancio 9-10? Perché ciò sia possibile deve uscire una sequenza composta da $ 0 \leq h \leq 8 $ lanci "croce" consecutivi e $ 9-h $ lanci "testa" consecutivi, in modo da ottenere una sequenza formata da $ C^hT^{9-h}C $. La probabilità è $ \p{C^hT^{9-h}C} = (0.45)^h \cdot (0.55)^{9-h} \cdot (0.45) = (0.45)^{h+1} \cdot (0.55)^{9-h}$. La probabilità dell'unione $ \bigcup_{h} $ delle stringhe sarà $ P = \sum_{h}(0.45)^{h+1}(0.55)^{9-h}$. Svolgere l'esercizio con una variabile geometrica.
	
%	TODO variabile geometrica
	
	\item Un ubriaco cammina in salita con probabilità $ \p{\text{salita}} = 1/4 $ oppure in discesa con probabilità $ \p{\text{discesa}} = 3/4 $. Ogni 10 secondi decide casualmente una direzione. Si muove lungo un asse $ X $ partendo dall'origine a velocità $ \dfrac{1m}{10s} $. Qual'è la posizione più probabile dopo $ 1 $ minuto? Introduciamo una variabile $ X = $ la posizione dopo 1 minuto. L'ubriaco si sposterà al massimo di 6 metri in salita o 6 metri in discesa, quindi $ X \in [-6, +6] $. Introduciamo anche la variabile $ Y = $ il numero di volte che l'ubriaco cambia direzione verso la discesa. $ Y $ è una variabile binomiale Bernoulliana (conta il numero di "successi" in 6 esperimenti ripetuti) $ \implies Y = B(6, 3/4) $. Abbiamo quindi che $ X = -1 \cdot Y + 1(6-Y) = 6 - 2Y $.
	\[ \implies P_k = \p{B(6, 3/4) = k} = \binom{6}{k} \left(\dfrac{3}{4}\right)^k \left(\dfrac{1}{4}\right)^{6-k} \]
	
	\begin{table}[H]
		\centering
		\caption{Distribuzione della variabile $ X $}
		\label{tab:ubriaco}
		\begin{tabular}{|l|l|l|l|l|l|l|l|}
			\hline \xrowht[()]{10pt}
			k     & 0                 & 1                        & 2                            & 3                            & 4                            & 5                           & 6                   \\ \hline \xrowht[()]{30pt}
			$P_k$ & $ \dfrac{1}{4^6}$  & $ \dfrac{6\cdot 3}{4^6}$ & $ \dfrac{15 \cdot 3^2}{4^6}$ & $ \dfrac{20 \cdot 3^3}{4^6}$ & $ \dfrac{15 \cdot 3^4}{4^6}$ & $ \dfrac{6 \cdot 3^5}{4^6}$ & $ \dfrac{3^6}{4^6}$ \\ \hline
		\end{tabular}
	\end{table}
	
\end{enumerate}

\subsection{Binomiale Negativa (o di Pascal)}
Sia data una Bernoulliana di parametro $ p $. Ripetiamo l'esperimento fino a che non ho $ n $ successi. Quanti sono i fallimenti ottenuti? Sappiamo che una variabile Binomiale conta i successi, una variabile Geometrica conta i fallimenti prima del primo successo e la Binomiale Negativa (NB) conta i fallimenti prima del successo $ n $-esimo. In una Binomiale Negativa non conta l'ordine degli esperimenti (tranne l'ultimo risultato). Le prove totali prima di avere $ n $ successi sono $ n+\text{NB} $. Sapendo che per avere $ n $ successi e $ k $ fallimenti, la probabilità di una Binomiale Negativa è definita come 

\[ \p{\nb{n, p} = k} = p^n(1-p)^k \] 

\paragraph{Esempio} 
Lancio una moneta fino ad ottenere 3 croci (non consecutive). Qual'è la probabilità di aver fatto esattamente 2 risultati testa? $ \p{\nb{3, \frac{1}{2}} = 2} = \binom{3+2-1}{2} \left(\frac{1}{2}\right)^3 \left(\frac{1}{2}\right)^2 = \frac{3}{16} $. 

La probabilità di ottenere almeno un risultato testa è $ \p{\nb{3, \frac{1}{2}} = 0} = 1 - \frac{1}{8} = \frac{7}{8} $

\section{Media Pesata, Speranza o Valore Atteso}
Consideriamo di voler calcolare la media pesata dei voti degli esami universitari. La media sarà per ogni esame $ i $:
\[ \sum_{i} \dfrac{(\text{voto})_i \cdot (\text{crediti})_i }{\sum_{i} \text{crediti}} = \sum_{i} (\text{voto})_i \cdot (\text{peso})_i \]

Definiamo $ X $ variabile aleatoria discreta $ X \subseteq \R $. La media di $ X $ è detta \textbf{speranza} o \textbf{valore atteso} e si denota con

\[ E[X] = \sum_{k} k \cdot \p{X = k} = \sum_{k} k \cdot p_k \].

Per calcolare la media di una funzione $ f : \R \to \R $:

\[ E[f(X)] = \sum_{k} f(k) \cdot p_k \]

\textbf{Osservazione} $\sum_{k} p_k = 1 $ \textbf{non implica che} $ \sum_{k} k p_k $ sia convergente. Se non converge ad un numero allora la variabile non ha media. Se la variabile assume solo valori positivi allora

\[  E[X] = \sum_{k>0} \p{X > k} = \sum_{k>0} (\p{X = k+1} + \p{X = k+2} + \dots) \]

\paragraph{Esercizi sulla speranza}

\begin{enumerate}
	\item Binomiale $ B(n,p) $. La distribuzione $ p_k = \binom{n}{k}p^k(1-p)^{n-k}  \forall k \in [0,n] $. Abbiamo che $ kp_k = k \binom{n}{k}p^k (1-p)^{n-k} = n \binom{n-1}{k-1} p^k(1-p)^{n-k} $. Otteniamo che 
		\[ E[B(n,p)] = \sum_{k=1}^{n} kp_k = \sum_{k=1}^{n} n \binom{n-1}{k-1} p^k(1-p)^{n-k} \]
		\[ = np \sum_{k} \binom{n-1}{k-1}p^{k-1}(1-p)^{n-k} \]
		Definiamo $ h = k-1 $
		\[=  np \sum_{h=0}^{n-1} \binom{n-1}{h} p^h(1-p)^{(n-1) - h } \]
		\[ = np(p+(1-p))^{n-1} = np \]
	\item Variabile geometrica $ \geom(p) $, conta il numero di successi prima di un successo:
		\[ E[\geom(p)] = \sum_{k=0}^{\infty} k(1-p)^k p = \sum_{k}\p{\geom(p) > k} \]
		Prendiamo in caso \[ \sum_{k}\p{\geom(p) > k} = \sum_{h=k+1}^{\infty} (1-p)^(h-1)p = \sum_{k}(1-p)^k = \dfrac{1}{1-(1-p)} = \dfrac{1}{p}\]
		
	\item Scommetto. Pago 1 euro. Lancio 3 dadi e guadagno 1 euro ogni 6 che esce.
	Rappresento il guadagno con una variabile binomiale $ X = B(3,1/6) - 1$. Abbiamo che $ E[X] = E[B(3,1/6)] - 1 = 3 \cdot 1/6 - 1 = -1/2$
	
	\item Prendo un seme di carte francesi $ \{ A, 2, \dots, 10, J, Q, K \} $
	L'asso ha valore 11. I numeri da 2 a 10 hanno lo stesso valore del numero, le figure hanno valore 10. Estraggo una carta. Definiamo una variabile aleatoria $ X = $ punteggio. $ X \in \{2,3,4,5,6,7,8,9,10,11\} $. Abbiamo che $ p_k = 0 \iff k< 2 \land k > 11 $. Abbiamo anche che $ p_k = 1/13 \iff k=2,\dots,11$ e $ p_k = 4/13 \iff k=10$
	  
\end{enumerate}

\subsection{Distribuzione di Poisson}
Nel caso di una variabile binomiale conosco $ p $ e $ n $ (numero esperimenti).
In una distribuzione di Poisson si conosce una media $ \mu $ di successi in un intervallo di osservazione.
Definiamo un intervallo $ \tau $, suddiviso in $ n $ sottointervalli. Abbiamo $ \mu $ successi. Se $ n $ è grande abbastanza tale che in ogni intervallo avviene 1 evento, generati da $ n $ esperimenti indipendenti allora $ \mu = E[B(n,p)] = np$. Ne segue che:
\[ p = \dfrac{\mu}{n} \]
\[ \p{B(n,p) = k} = \binom{n}{k}p^k(1-p)^{n-k} = \binom{n}{k} \left(\dfrac{\mu}{n}\right)^k \left(1- \dfrac{\mu}{n}\right)^{n-k} \]
\[ = \dfrac{n!}{k!(n-k)!} \dfrac{1}{n^k} \dfrac{\mu^k}{\left(1-\dfrac{\mu}{n}\right)^k} \cdot \left(1-\dfrac{\mu}{n}\right)^n \] 
\[ = \dfrac{n^k \cdot \left(1-\dfrac{1}{n} \right) \cdot \left(1-\dfrac{2}{n}\right) \cdot \dots \cdot \left(1- \dfrac{k-1}{n}\right)}{k!} \cdot \dfrac{\mu^k}{\left(1-\dfrac{\mu}{n}\right)^k} \cdot \left(1-\dfrac{\mu}{n}\right)^n \]

Ne otteniamo che 
\[ \p{\text{Poisson}(\mu) = k} = \dfrac{\mu^k}{k!} e^{-\mu} \]

% TODO dimostrazione strana

\paragraph{Esempio}
Siamo nel secolo 1800, prendiamo l'esercito di Napoleone nel reparto della cavalleria. Ogni anno 12 cavalieri muoiono per incidente a cavallo. Voglio sapere la probabilità che nel 1861 siano morti 7 cavalieri.
\[ \p{\text{anno } 1861 | \text{sono morti 7 cavalieri}} \]
Utilizziamo la distribuzione di Poisson.

\[ \p{\text{Poisson}(12) = 7} = \dfrac{12^7}{7}e^{-12} \approx 0.04  \]


\paragraph{Linearità della media}

Siano date $ X,Y $ variabili aleatorie, $ \alpha ,\beta \in \R $. Abbiamo che
\[ E[\alpha X + \beta Y] = \alpha E[X] + \beta E[Y] \]

%TODO boh

\paragraph{Valore atteso del prodotto}
Siano date $ X,Y $ variabili aleatorie \textbf{indipendenti}.
\[ E[X \cdot Y] = E[X] \cdot E[Y] = \sum_{i,j} i \cdot j \cdot \p{X = i \cap Y =j}\]
\[ = \sum_{i,j} i \cdot j p_{ij} = \sum_{i,j} ij p_i^X p_j^Y =  \sum_{i} i p_i^X \cdot \sum p_j^Y\]